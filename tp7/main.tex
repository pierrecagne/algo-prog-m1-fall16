% !TEX program = lualatex
\documentclass[french,a4paper]{article}

\usepackage{amsmath,amsthm,amssymb}
\usepackage{fontspec}
% \usepackage{unicode-math}
%\defaultfontfeatures{Ligatures=TeX}
\setmainfont[Ligatures=TeX]{Cambria}
\setsansfont[Ligatures=TeX]{Calibri}
%\setmathfont{Cambria Math}
\setmonofont{Consolas}[Scale=MatchLowercase]

\usepackage{stmaryrd}
\usepackage{polyglossia}
\usepackage{enumitem}
\usepackage{tikz,tikz-cd}
\usepackage[explicit]{titlesec}
\usepackage{listings}
\usepackage{hyperref}
\usepackage{fmtcount}
\usepackage{microtype} % uncomment for better rendering

\setmainlanguage{french}

\title{\sffamily TP7: manipulation de fichiers, modularisation du code}%
\date{15 novembre 2016}%
\author{Algorithmique et projet de programmation --- M1
  mathématiques}%

\lstdefinestyle{sh}{%
  language=sh,%
  basicstyle=\tt\small,%
}
\lstdefinestyle{C}{%
  language=C,%
  %breaklines=true,%
  frame=l,%
  xleftmargin=\parindent,%
  basicstyle=\ttfamily\small,%
  keywordstyle=\bfseries\color{green!40!black!80},%
  showstringspaces=false,%
  commentstyle=\itshape\color{purple!70},%
  identifierstyle=\color{blue!80},%
  stringstyle=\color{red!80},%
  directivestyle=\color{orange!90!black!80},%
  upquote=true,%
  otherkeywords={stack_t,queue_t},%
  escapeinside={<latex>}{</latex>},%
}
\lstset{style=C}

\theoremstyle{definition}
\newtheorem{exercise}{Exercice}
\theoremstyle{remark}
\newtheorem*{remark}{Remarque}

\newcommand{\shell}[1]{\lstinline[style={},style=sh]|#1|}
\newcommand{\inlinec}[1]{\lstinline[style=C]°#1°}

\begin{document}
\maketitle

\section{Commande {\tt sort}}
\label{sec:sort-unix}

On rappelle qu'on essaye de recoder la commande \shell{sort} des
sytèmes UNIX, i.e.\ programme qui prend en paramètre un chemin d'accès
à un fichier texte et qui affiche à l'écran l'ensemble de ses lignes
triées lexicographiquement.

Lors du TP dernier, on a écrit deux fonctions:
\begin{lstlisting}
char compare_str (char* s, char* t);
void sort_str (char** strs);
\end{lstlisting}
La première compare deux chaînes de caratère pour l'ordre
lexicographique, la deuxième trie (selon votre méthode péférée) un
tableau de chaînes de caractères.

Il s'agit maintenant d'utiliser ces fonctions pour coder un programme
complet imitant \shell{sort}. La bibliothèque \inlinec{stdio.h}
contient les fonctions suivantes afin de lire dans un fichier:
\lstinputlisting{fopen.c}

\begin{exercise}
  \label{ex:sort}%
  \'Ecrire un programme imitant la commande \shell{sort}, qu'on
  appellera en ligne de commande comme suit:
  \begin{lstlisting}[style={},style=sh]
./mysort path/to/my/file
  \end{lstlisting}
  où \shell{path/to/my/file} est un chemin d'accès à un fichier texte.

  En particulier, le programme commencera par déterminer le nombre de
  lignes et la taille maximale des lignes du fichier passé en
  paramètre pour allouera de la mémoire en quantité suffisante, avant
  d'appeler la fonction \inlinec{sort_str}.
\end{exercise}

\section{Votre première bibliothèque}
\label{sec:own-string}

Dans la suite, on va apprendre à {\em modulariser} son code. Jusqu'à
maintenant, on a toujours mis tout notre code dans un même fichier:
imaginer faire cela pour un projet de plusieurs milliers de ligne de
code... Ce serait un bazar sans nom! Heureusement, le langage {\tt C}
nous autorise à partitionner notre code en différents fichiers
communiquant entre eux. En fait, vous utilisez déjà cette
fonctionnalité du langage quand vous incluez les bibliothèques
\inlinec{stdio.h} et \inlinec{stdlib.h}.

Au-delà de l'organisation, cette technique permet aussi d'{\em
  encapsuler} du code: c'est-à-dire de proposer à l'utilisateur (ici
un autre codeur) une interface pour manipuler des données sans qu'il
ait à en comprendre l'implémentation. Vous avez vous-même déjà
expérimenté l'encapsulation en tant qu'utilisateur, pas plus tard qu'à
l'exercice \ref{ex:sort}, dans la manipulation de fichier: vous n'avez
pas besoin de connaître tous les détails de la structure
\inlinec{FILE} car vous avez accès à l'interface que sont les
fonctions \inlinec{fopen}, \inlinec{fgets}, etc.

Enfin, c'est un bon moyen de maintenir une certaine {\em
  rétrocompatibilité}. Si demain les créateurs du langage {\tt C}
change l'implémentation de la structure \inlinec{FILE} (pour s'adapter
eux-même à un changement de norme dans les systèmes de fichiers par
exemple), votre programme \shell{sort} continuera de compiler sans
souci du moment que les prototypes et comportements des fonctions
\inlinec{fopen}, \inlinec{fgets}, etc. restent les mêmes. En revanche,
si vous aviez utilisé les champs spécifiques de la structure
\inlinec{FILE}, votre programme pourrait tout à fait être rendu
obsolète par une telle mise à jour des créateurs de {\tt C}.

\medskip

Il est temps de passer du côté développeur de l'encapsulation! Vous
avez sûrement remarqué que l'usage des chaînes de caractères en {\tt
  C} est très pénible: tous les rouages de leur utilisations sont
exposés à l'utilisateur, laissant libre cours à une gestion
schyzophrénique de la mémoire, ce qui résulte généralement en tout un
tas de \shell{segfault}... On va donc créer une petite bibliothèque
personnel, nommée \shell{proper_string}, qui va gérer tous les
problèmes de mémoire en arrière-plan et laisser à l'utilisateur
seulement l'usage normale des chaînes de caractères.

\begin{exercise}
  Créer un fichier \shell{proper_string.c} contenant la déclaration
  d'une structure:
  \begin{lstlisting}
struct string_s {
  int length; // maintain the length of the string
  char *val; // actual string, without the now useless trailing '\0'
} string_t;
// let's now call a proper string avery variable of type string_t
  \end{lstlisting}
  
  \'Ecrire dans ce même fichier les fonctions de manipulation de
  chaînes de caractères que vous jugez utiles, e.g.:
  \begin{lstlisting}[breaklines=true]
/* All functions get a 'ps_' prefix to recall that it is part of the 'proper_string' library' */
int ps_length (string_t str); // gives back the length of the proper string
string_t ps_init (const char* str); // transforms an actual C string into a proper string
string_t ps_empty (); // creates an empty proper string; should be equivalent to ps_init("")
string_t ps_copy (string_t str); // returns a copy of the proper string str
string_t ps_concat (string_t fst, string_t snd); // gives back the concatenation of two proper strings
void ps_print (string_t str); // prints the proper string
char *ps_toC (string_t str); // transform a proper string into a regular C string, i.e. a char array with trailing '\0'

/* ... whatever you could have wanted when using regular strings ...*/
  \end{lstlisting}
\end{exercise}

On a donc dans le fichier \shell{proper_string.c} une implémentation
des chaînes de caractères et des fonctions de bases permettant de
manipuler celles-ci. Ce n'est d'ailleurs pas la seule implémentaiton
possible! Vous pourriez avoir des raisons de choisir une autre
structure de données pour le champ \inlinec{val}: liste chaînées ou
doublement chaînées, une table de hashage, etc. Ou bien vous pourriez
vouloir ajouter d'autres champs: un entier \inlinec{nb_spaces} pour le
nombre d'espaces, un booléen \inlinec{all_num} pour savoir si tous les
caractères sont numériques, etc. C'est à vous de voir, en fonction de
l'utilisation que vous voulez en faire!

Il va falloir maintenant mettre en place l'interface, ce à quoi
l'utilisateur aura accès: cela se fait par l'intermédiaire d'une {\em
  header}.
\begin{exercise}
  Créer un fichier \shell{proper_string.h} contenant les prototypes
  des fonctions et des structures auxquelles vous voulez donner accès
  à l'utilisateur.

  Il faut alors penser à l'inclure dans \shell{proper_string.c}:
  c'est-à-dire qu'il faut ajouter au préprocessing une nouvelle ligne
  \begin{lstlisting}
#include "proper_string.h" // beware, it is quotes, not chevrons
  \end{lstlisting}
\end{exercise}

\begin{remark}
  Remarquez bien qu'il n'y a aucune fonction \inlinec{main} dans le
  fichier \shell{proper_string.c}. C'est normal car on ne veut pas
  exécuter directement du code de ce fichier, mais seulement s'en
  servir comme d'une banque de fonctions.
\end{remark}

On va maintenant utiliser cette bibliothèque.
\begin{exercise}
  \'Ecrire un programme dans un fichier \shell{prog.c} (vou pouvez
  bien sûr utiliser le nom que vous souhaitez pour le source) qui
  prend deux chaînes de caractères en paramètres en ligne de commande
  et qui en fait la concaténation sous forme de \inlinec{string_t} et
  qui affiche sa longueur.

  Il faudra en particulier inclure \shell{proper_string.h}, et on
  compilera en rajoutant le fichier annexe \shell{proper_string.c}
  en argument de \shell{gcc}:
  \begin{lstlisting}[style={},style=sh]
gcc -Wall -o myprog prog.c proper_string.c 
  \end{lstlisting}
\end{exercise}


\section{Pour le 22 novembre}
\label{sec:homeworks}

\begin{exercise}
  Recoder la commande {\tt sort} mais en utilisant votre nouvelle
  bibliothèque de strings. On n'hésitera pas à y ajouter les fonctions
  nécessaires.
\end{exercise}

\end{document}