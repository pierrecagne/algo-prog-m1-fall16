% !TEX program = lualatex
\documentclass[french,a4paper]{article}

\usepackage{amsmath,amsthm,amssymb}
\usepackage{fontspec}
\usepackage{unicode-math}
\defaultfontfeatures{Ligatures=TeX}
\setmainfont{Cambria}
\setsansfont{Calibri}
%\setmathfont{Cambria Math}
\setmonofont{Consolas}[Scale=MatchLowercase]

\usepackage{stmaryrd}
\usepackage{polyglossia}
\usepackage{enumitem}
\usepackage{tikz,tikz-cd}
\usepackage[explicit]{titlesec}
\usepackage{listings}
\usepackage{hyperref}
\usepackage{fmtcount}
\usepackage{microtype} % uncomment for better rendering

\setmainlanguage{french}

\title{\sffamily TP2: un peu plus de {\tt C}}
\date{27 septembre 2016}
\author{Algorithmique et projet de programmation --- M1 mathématiques} 

\lstdefinestyle{sh}{%
  language=sh,%
  basicstyle=\tt\small,%
}
\lstdefinestyle{C}{%
  language=C,%
  breaklines=true,%
  frame=l,%
  xleftmargin=\parindent,%
  basicstyle=\ttfamily\small,%
  keywordstyle=\bfseries\color{green!40!black!80},%
  showstringspaces=false,%
  commentstyle=\itshape\color{purple!70},%
  identifierstyle=\color{blue!80},%
  stringstyle=\color{red!80},%
  directivestyle=\color{orange!90!black!80},%
  % otherkeywords={},%
  escapeinside={<latex>}{</latex>},%
}
\lstset{style=C}

\theoremstyle{definition}
\newtheorem{exercise}{Exercice}
\theoremstyle{remark}
\newtheorem*{remark}{Remarque}

\newcommand{\shell}[1]{\lstinline[style=sh]|#1|}
\newcommand{\inlinec}[1]{\lstinline[style=C]°#1°}

\begin{document}
\maketitle

\section{Comme de par hasard}
\label{sec:random}

La bibliothèque \inlinec{stdlib.h} contient une fonction qui génère un
nombre (pseudo)aléatoire: \inlinec{rand()} renvoie un entier aléatoire
entre \inlinec{0} et une (grande) valeur abitraire fixée par {\tt C}
et appelée \inlinec{RAND_MAX}. Pour que \inlinec{rand} se comporte
réellement comme un génrateur aléatoire, il faut l'initialiser avec
une {\em graine}, par exemple l'heure: \lstinputlisting{random.c}

\begin{exercise}
  \'Ecrire un programme qui génère un nombre aléatoire entre 0 et un
  entier \inlinec{n} fixé (exclu).
\end{exercise}

\begin{exercise}
  \label{exo:simulate-42}%
  On peut vérifier expérimentalement si le générateur pseudo-aléatoire
  est correct: écrire un programme qui tire un millions de fois un
  nombre au hasard entre \inlinec{0} et \inlinec{100} et qui affiche
  la proportion de \inlinec{42} parmi ce tirage.
\end{exercise}

\begin{exercise}
  \'Ecrire un jeu du {\em plus ou moins}. Le jeu se déroule comme suit:
  \begin{enumerate}[label=(\arabic*)]
  \item un nombre \inlinec{bound} est fixé à l'avance,
  \item un nombre de coups \inlinec{chances} est fixé à l'avance,
  \item le programme génère un nombre aléatoirement entre \inlinec{0}
    et \inlinec{bound},
  \item l'utilisateur est invité à entrer un nombre et le programme
    indique si le nombre cherché est plus grand ou plus petit (ou
    égal),
  \item on répète cette instruction jusqu'à ce que le joueur trouve le
    nombre (et dans ce cas il gagne) ou que le nombre de coups soit
    écoulé (et dans ce cas il perd).
  \end{enumerate}
\end{exercise}

\begin{exercise}
  \'Echangez les rôles. C'est-à-dire que c'est maintenant
  l'utilisateur qui choisit le nombre mystère puis l'ordinateur qui
  essaye de le deviner: à chaque tour, l'ordinateur propose un nombre
  et le joueur répond \inlinec{1} si le nombre mystère est plus grand,
  \inlinec{-1} s'il est plus petit, et enfin \inlinec{0} si
  l'ordinateur a trouvé le nombre.

  Faites utiliser une stratégie ``bêbête'' à l'ordinateur : à chaque
  tour, il tire un nombre au hasard (dans l'intervalle) .
\end{exercise}

\begin{exercise}
  Améliorer la stratégie de l'ordinateur:
  \begin{enumerate}[label=(\arabic*)]
  \item tout d'abord en le faisant choisir un nombre au hasard dans un
    intervalle plus malin qu'il mettra à jour avec les informations
    donnés par l'utilisateur à chaque tour,
  \item puis en lui faisant utiliser la dichotomie: comme
    précédemment, on met à jour l'intervalle à chaque tour, et on y
    choisit non plus un élément au hasard mais le milieu pour réduire
    au maximum la taille de l'intervalle au tour d'après.
  \end{enumerate}
\end{exercise}

\section{Des arguments}
\label{sec:args}

L'avantage d'appeler un programme/logiciel en ligne de commande plutôt
qu'en cliquant sur une icône est la possibilité de passer des {\em
  arguments} au dit programme: c'est-à-dire des valeurs que le
programme pourra utiliser lors de son exécution. Par exemple,
\shell{firefox "www.google.com"} ouvre Mozilla Firefox directement à
la page de recherche Google au lieu de la page d'accueil par défaut.

On peut le faire également pour nos programmes écrits en {\tt C}:
c'est le rôle des deux variables mystérieuses \inlinec{argc} et
\inlinec{argv} que je vous incite à placer en arguments de
\inlinec{main}. La variable \inlinec{argc} de type \inlinec{int}
contient le nombre d'arguments passés au programme tandis que
\inlinec{argv} de type \inlinec{char**} (tableau de chaînes de
caractères) contient les arguments eux-même, sous forme de chaînes de
caractères. Pour accéder à l'argument numéro $i$, on écrit simplement
\inlinec{argv[i]}. Attention: la numérotation commence à \inlinec{0}
(et se finit donc à \inlinec{argc-1}), et la première case
\inlinec{argv[0]} contient toujours le nom du programme lui-même (en
particulier, \inlinec{argc} est toujours supérieur ou égal à 1).

\begin{exercise}
  Réécrire le programme du plus ou moins en laissant l'utilisateur
  passer deux arguments en ligne de commande: le premier sera la borne
  suprieure (appelée \inlinec{bound} précédemment) et le deuxième le
  nombre de coups (appelée \inlinec{chances} précédemment). Indice: il
  faudra utiliser la fonction \inlinec{sscanf(s,format,ref)} qui
  fonctionne comme \inlinec{scanf} mais prend comme entrée non pas le
  clavier mais la chaîne \inlinec{s} passée en premier paramètre. Par
  exemple, \inlinec{sscanf(s,"\%d",\&var)} stocke \inlinec{10} dans la
  variable \inlinec{var} si la chaîne \inlinec{s} vaut \inlinec{"10"}.
\end{exercise}

\section{Introduction aux fonctions}
\label{sec:function}

L'usage des fonctions en {\tt C}, et de manière générale en
programmation, est double: factoriser et modulariser le code. La {\em
  factorisation de code} consiste à éviter les répétitions bêtes et
méchantes de code. Prenons l'exemple d'un programme mettant en jeu
trois entiers, et qui à un moment donné doit vérifier si ces trois
entiers sont des carrés parfaits: \lstinputlisting{perfectsq.c} On
aimerait pouvoir n'écrire qu'une seule fois les instructions
déterminant si un nombre est un carré parfait puis l'appliquer à
\inlinec{a}, \inlinec{b} et \inlinec{c} indifféremment. C'est
exactement ce que réalise une fonction {\tt C}:
\lstinputlisting{perfectsq_fun.c} Mais les fonctions peuvent aussi
simplement servir à {\em modulariser} le code. C'est la pratique qui
consiste à garder séparé dans le code les parties du programme que
l'on {\em pense séparément}. On illustrera cela tout au long du
semestre.

Une fonction s'instancie sous la forme suivante:
\begin{lstlisting}
return_type function_name(arg_type1 arg1, arg_type2 arg2, ...) {
  /* code with one or more return of the type return_type */
}  
\end{lstlisting}
Vous avez en fait déjà codé des fonctions {\tt C}. En effet, tous vos
sources contiennent une fonction \inlinec{main}, où le type de retour
est \inlinec{int}, et qui prend deux arguments, le premier
\inlinec{argc} de type \inlinec{int} et le second \inlinec{argv} de
type \inlinec{char**}. La fonction précédente \inlinec{issquare} a
quat à elle pour type de retour \inlinec{int} et prend un unique
argument \inlinec{n} de type \inlinec{int}.

Utiliser une fonction dans votre code est aussi naturel qu'appeler une
fonction en mathématiques:
\begin{lstlisting}
function_name(val1, val2, ...)
\end{lstlisting}
\`A ceci près que \inlinec{val1} doit être de type
\inlinec{arg_type1}, \inlinec{val2} de type \inlinec{arg_type2},
etc. De même, l'expression \inlinec{function_name(val1,val2,...)} doit
être utilisée là où est valide un objet de type
\inlinec{return_type}. Encore une fois, ceci n'est pas une nouveauté,
vous utilisez les fonctions {\tt C} depuis maintenant deux
semaines. La fonction \inlinec{printf}, \inlinec{scanf},
\inlinec{rand}, etc.

\begin{exercise}
  \'Ecrire une {\bf fonction} de prototype:
  \begin{lstlisting}
float simulate(int n, int k, int inf, int sup)
  \end{lstlisting}
  Cette fonction doit simuler le tirage répété $n$ fois d'un entier
  dans un intervalle (d'entiers) $[\mathrm{inf},\mathrm{sup}[$ donné
  et retourner la fréquence d'apparition de $k$ dans ce tirage.
  
  Ainsi le programme de l'exercice \ref{exo:simulate-42} doit
  correspondre au bout de code suivant:
  \begin{lstlisting}
printf("42 appears %f%c of the time.\n",
        simulate(1000000, 42, 0, 100), '%');
  \end{lstlisting}
\end{exercise}

\begin{exercise}
  Améliorer légèrement le programme précédent pour permettre à
  l'utilisateur de passer en paramètres en ligne de commande le nombre
  de tirage, le nombre test duquel on renvoie la fréquence, et les
  bornes de l'intervalle de test.
\end{exercise}

\section{\`A rendre pour le 4 octobre 2016}
\label{sec:homework}

\begin{exercise}
  \'Ecrire une {\bf fonction} \inlinec{prime} de prototype:
  \begin{lstlisting}
int prime(unsigned int n)
  \end{lstlisting}
  qui prend en argument un entier $n$ positif et renvoie \inlinec{0}
  s'il n'est pas premier et \inlinec{1} s'il l'est.

  Utiliser cette fonction dans un programme qui affiche la proportion
  de nombre premier dans l'intervalle $[0,n]$ pour un nombre $t$ de
  tirage où $n$ et $t$ sont passés en paramètres en ligne de commande
  par l'utilisateur.
\end{exercise}

\begin{exercise}
  Pour tout $m \in \mathbb N$, on définit une suite
  $(s^m_n)_{n\geq 0}$ donnée par
  \begin{displaymath}
    s^m_0 = m \quad \text{et} \quad
    \forall n>0,\, \left\{
      \begin{aligned}
        s^m_n &= \frac{s^m_{n-1}} 2
        \quad \text{si $s^m_{n-1}$ est pair} \\
        s^m_n &= 3{s^m_{n-1}}+1 \quad \text{sinon}
      \end{aligned}
    \right.
  \end{displaymath}
  La conjecture de Syracuse affirme que pour tout $m \in \mathbb N$,
  la suite $(s^m_n)_{n\geq 0}$ stagne en $1$. \'Ecrire un programme
  qui vérifie la conjecture de Syracuse pour $m \in [0,100]$.
\end{exercise}

\end{document}