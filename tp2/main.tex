% !TEX program = lualatex
\documentclass[french,a4paper]{article}

\usepackage{amsmath,amsthm,amssymb}
\usepackage{fontspec}
\usepackage{unicode-math}
\defaultfontfeatures{Ligatures=TeX}
\setmainfont{Cambria}
\setsansfont{Calibri}
%\setmathfont{Cambria Math}
\setmonofont{Consolas}[Scale=MatchLowercase]

\usepackage{stmaryrd}
\usepackage{polyglossia}
\usepackage{enumitem}
\usepackage{tikz,tikz-cd}
\usepackage[explicit]{titlesec}
\usepackage{listings}
\usepackage{hyperref}
\usepackage{fmtcount}
\usepackage{microtype} % uncomment for better rendering

\setmainlanguage{french}

\title{\sffamily TP2: un peu plus de {\tt C}}
\date{20 septembre 2016}
\author{Algorithmique et projet de programmation --- M1 mathématiques} 

\lstdefinestyle{sh}{%
  language=sh,%
  basicstyle=\tt\small,%
}
\lstdefinestyle{C}{%
  language=C,%
  breaklines=true,%
  frame=l,%
  xleftmargin=\parindent,%
  basicstyle=\ttfamily\small,%
  keywordstyle=\bfseries\color{green!40!black!80},%
  showstringspaces=false,%
  commentstyle=\itshape\color{purple!70},%
  identifierstyle=\color{blue!80},%
  stringstyle=\color{red!80},%
  directivestyle=\color{orange!90!black!80},%
  % otherkeywords={},%
  escapeinside={<latex>}{</latex>},%
}
\lstset{style=C}

\theoremstyle{definition}
\newtheorem{exercise}{Exercice}
\theoremstyle{remark}
\newtheorem*{remark}{Remarque}

\newcommand{\shell}[1]{\lstinline[style=sh]|#1|}
\newcommand{\inlinec}[1]{\lstinline[style=C]°#1°}

\begin{document}
\maketitle

\section{Comme de par hasard}
\label{sec:random}

La bibliothèque \inlinec{stdlib.h} contient une fonction qui génère un
nombre (pseudo)aléatoire: \inlinec{rand()} renvoie un entier aléatoire
entre \inlinec{0} et une (grande) valeur abitraire fixée par {\tt C}
et appelée \inlinec{RAND_MAX}. Pour que \inlinec{rand} se comporte
réellement comme un génrateur aléatoire, il faut l'initialiser avec
une {\em graine}, par exemple l'heure: \lstinputlisting{random.c}

\begin{exercise}
  \'Ecrire un programme qui génère un nombre aléatoire entre 0 et un
  entier \inlinec{n} fixé (exclu).
\end{exercise}


\begin{exercice}
  
\end{exercice}

\end{document}