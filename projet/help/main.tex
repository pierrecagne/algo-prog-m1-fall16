% !TEX program = lualatex
\documentclass[french,10pt]{beamer}

\usepackage{amsmath,amsthm,amssymb}
\usepackage{fontspec}
%\usepackage{unicode-math}
\defaultfontfeatures{Ligatures=Common}
\setmainfont{Cambria}
\setsansfont{Calibri}
\setmonofont{Consolas}[Scale=MatchLowercase]

\usepackage{stmaryrd}
\usepackage{polyglossia}
\usepackage{enumitem}
\usepackage{tikz,tikz-cd}
%\usepackage[explicit]{titlesec}
\usepackage{listings}
\usepackage{hyperref}
\usepackage{fmtcount}
\usepackage{microtype} % uncomment for better rendering

\usetikzlibrary{decorations.markings}

\setmainlanguage{french}
\setlist[itemize]{label=--}

\lstdefinestyle{sh}{%
  language=sh,%
  basicstyle=\tt\small,%
}
\lstdefinestyle{C}{%
  language=C,%
  breaklines=true,%
  %frame=l,%
  xleftmargin=\parindent,%
  basicstyle=\ttfamily\footnotesize,%
  keywordstyle=\bfseries\color{green!40!black!80},%
  showstringspaces=false,%
  commentstyle=\itshape\color{purple!70},%
  identifierstyle=\color{blue!80},%
  stringstyle=\color{red!80},%
  directivestyle=\color{orange!90!black!80},%
  % otherkeywords={},%
  % escapeinside={<latex>}{</latex>},%
}
\lstset{style=C}


\title{UGMT41: aide au projet}
\date{}
\author{Pierre Cagne}

\newcommand{\shell}[1]{\lstinline[style={},style=sh]|#1|}
\newcommand{\inlinec}[1]{\lstinline[style=C]°#1°}

\DeclareMathOperator{\powersetoperator}{\mathcal P}
\newcommand{\powerset}[1]{\powersetoperator(#1)}

\tikzset{frame/.style = {%
    rounded corners,%
    draw = black!20,%
    line width = 1pt,%
    fill = black!5
  },%
  translate/.style = {%
    ->,%
    line width = 7pt,%
    red!60,%
    >=stealth
  }%
}

\def\possible{\color{structure.fg!60!black!70}}
\def\constraint{\color{structure.fg!50!red!70}}
\def\satisfaction{\color{structure.fg!60!green!70}}
\def\solution{\color{red!70}}
\def\property{\color{green!60!black!70}}

\begin{document}

\maketitle

\begin{frame}{Description du jeu de sudoku}
  \begin{itemize}
  \item {\color{structure.fg} Données:} une grille 9 par 9 (avec
    certaines cases déjà remplies)
  \item {\possible Remplissage:} chaque case peut contenir
    un chiffre $1,2,\dots$ou $9$.
  \item {\constraint Règles:}
    \begin{enumerate}[label=\null]
    \item toute case doit contenir un chiffre
    \item chaque colonne contient un $1$, un $2$, $\dots$et un $9$
    \item chaque ligne contient un $1$, un $2$, $\dots$et un $9$
    \item chaque bloc contient un $1$, un $2$, $\dots$et un $9$
    \end{enumerate}
  \item {\solution Solution}: la sélection d'un remplissage
    de chaque case, de sorte que chacune des règles est satisfaite une
    unique fois.
  \end{itemize}
\end{frame}

\begin{frame}{Première traduction}
  \begin{tikzpicture}[overlay,remember picture]
    \node [frame,anchor=west,xshift=1em] (left) at (current page.west)
    {%
      \begin{minipage}{.49\linewidth}
        \begin{itemize}
        \item Problème: une {\em grille} avec
          \begin{enumerate}[label=,leftmargin=5pt]\small
          \item \possible remplissages possibles de cases,
          \item \constraint règles
          \end{enumerate}
          \vskip 20pt%
        \item Solution: une {\em sélection} de
          \begin{center}
            \solution \small remplissages pour chaque case
          \end{center}
          tels que
          \begin{center} \property%
            toute règle est satisfaite une unique fois
          \end{center}
        \end{itemize}
      \end{minipage}
    };%
    \node [frame,anchor=east,xshift=-1em] (right) at (current
    page.east) {%
      \begin{minipage}{.49\linewidth}
        \begin{itemize}
        \item Problème: un {\em triplet}
          \begin{displaymath}
            ({\possible P}, {\constraint C}, {\satisfaction \mathcal S})
          \end{displaymath}
        \item Solution: une {\em sélection}
          \begin{displaymath}
            \color{red!70} %
            p_0, p_1, \dots, p_n
          \end{displaymath}
          d'élément de $P$ tels que
          \begin{center}
            \color{green!60!black!70}%
            toute contrainte est satisfaite par un unique $p_i$
          \end{center}
        \end{itemize}
      \end{minipage}
    };%
    %
    \draw [translate] (left.east) to (right.west);
  \end{tikzpicture}
\end{frame}

\begin{frame}{Deuxième traduction}
  \begin{tikzpicture}[overlay,remember picture]
    \node [frame,anchor=west,xshift=1em] (left) at (current page.west)
    {%
      \begin{minipage}{.49\linewidth}
        \begin{itemize}
        \item Problème: un {\em triplet}
          \begin{displaymath}
            ({\possible P}, {\constraint C}, {\satisfaction \mathcal S})
          \end{displaymath}
        \item Solution: une {\em sélection}
          \begin{displaymath}
            \solution %
            p_0, p_1, \dots, p_n
          \end{displaymath}
          d'élément de $P$ tels que
          \begin{center}
            \property %
            toute contrainte est satisfaite par un unique $p_i$
          \end{center}
        \end{itemize}
      \end{minipage}
    };%
    \node [frame,anchor=east,xshift=-1em] (right) at (current
    page.east) {%
      \begin{minipage}{.49\linewidth}
        \begin{itemize}
        \item Problème: une {\em matrice}
          \begin{displaymath} %
            M = (m_{{\possible i},{\constraint j}})\quad%
            m_{{\possible i},{\constraint j}} \in {\satisfaction \{ 0,1 \}}
          \end{displaymath}
        \item Solution: une {\em sélection} 
          \begin{displaymath}
            \solution %
            \ell_0, \ell_1, \dots, \ell_n
          \end{displaymath}
          de lignes de $M$ tels que
          \begin{center}
            \property %
            toute colonne contient un unique $1$ parmi les lignes
            $\ell_i$
          \end{center}
        \end{itemize}
      \end{minipage}
    };%
    %
    \draw [translate] (left.east) to (right.west);
  \end{tikzpicture}
\end{frame}

\begin{frame}{Algorithme récursif}
  \begin{tikzpicture}[overlay,remember picture]
    \node [frame,anchor=west,xshift=1em] (left) at (current page.west)
    {%
      \begin{minipage}{.49\linewidth}
        Une solution
        \begin{displaymath}
          \solution %
          p_0, p_1, \dots, p_n
        \end{displaymath}
        de $({\possible P},{\constraint C},{\satisfaction\mathcal S})$
        est exactement la donnée
        \begin{itemize}
        \item de ${\constraint c \in C}$ et ${\solution p_i}$ avec
          $p_i \mathrel{\mathcal S} c$
        \item et d'une solution
          \begin{displaymath}
            \solution %
            p_0,p_1,\dots,\widehat{p_i},\dots,p_n
          \end{displaymath}
          de
          $({\possible P'}, {\constraint C'}, {\satisfaction \mathcal
            S})$ où
          \begin{enumerate}[label=,leftmargin=0pt]
          \item
            $\possible P' = \{p : \forall c, p_i \mathrel{\mathcal S}c
            \mathrel\Rightarrow p \mathrel{\not \kern-2pt\mathcal S} c
            \}$
          \item $\constraint C' = \{c : p \mathrel{\not \kern-2pt\mathcal S} c \}$ 
          \end{enumerate}
        \end{itemize}
      \end{minipage}
    };%
    \node [frame,anchor=east,xshift=-1em] (right) at (current
    page.east) {%
      \begin{minipage}{.49\linewidth}
        Une solution
        \begin{displaymath}
          \solution %
          \ell_0, \ell_1, \dots, \ell_n
        \end{displaymath}
        de $M =(m_{i,j})$ est exactement la donnée
        \begin{itemize}
        \item de ${\constraint k}$ et ${\solution \ell_i}$ avec
          $m_{\ell_i,k} = 1$
        \item et d'une solution
          \begin{displaymath}
            \solution %
            \ell_0,\ell_1,\dots,\widehat{\ell_i},\dots,\ell_n
          \end{displaymath}
          de la sous-matrice $M'$ de $M$ après
          \begin{enumerate}[label=,leftmargin=0pt]\small%
          \item \possible suppression des lignes ayant un $1$ au même
            colonne que la ligne $\ell_i$
          \item \constraint suppression des colonnes ayant un $1$ dans
            la ligne $\ell_i$
          \end{enumerate}
        \end{itemize}
      \end{minipage}
    };%
    %
    \draw [translate] (left.east) to (right.west);
  \end{tikzpicture}
\end{frame}

\begin{frame}[fragile]{Un mot sur l'implémentation}
  \begin{tikzpicture}[overlay,remember picture]
    \node [frame,anchor=west,xshift=1em] (left) at (current page.west)
    {%
      \begin{minipage}{.55\linewidth}
        \centering%
        \scalebox{.5}{%
          \begin{tikzpicture}[pointers/.style={%
              postaction={decorate},decoration={markings, mark=at position
                0.8 with {\arrow{>}}},shorten >=4pt%
            },%
            box/.style={solid,thin},%
            every node/.style={minimum height=2em,minimum width=2em}%
            ]
            %% create nodes
            \node (root) at (0,0) {$\phantom{C_1}$};%
            \node (c1) at (2,0) {$C_1$};%
            \node (c2) at (4,0) {$C_2$};%
            \node (c3) at (6,0) {$C_3$};%
            \node (c4) at (8,0) {$C_4$};%
            \node (c5) at (10,0) {$C_5$};%
            \node (c12) at (2,-4) {\color{lightgray} 1};%
            \node (c14) at (2,-8) {\color{lightgray} 1};%
            \node (c21) at (4,-2) {\color{lightgray} 1};%
            \node (c23) at (4,-6) {\color{lightgray} 1};%
            \node (c33) at (6,-6) {\color{lightgray} 1};%
            \node (c35) at (6,-10) {\color{lightgray} 1};%
            \node (c42) at (8,-4) {\color{lightgray} 1};%
            \node (c51) at (10,-2) {\color{lightgray} 1};%
            \node (c52) at (10,-4) {\color{lightgray} 1};%
            \node (c54) at (10,-8) {\color{lightgray} 1};%
            %% box them and base for pointers
            \foreach \x in
            {root,c1,c2,c3,c4,c5,c12,c14,c21,c23,c33,c35,c42,c51,c52,c54} {%
              \fill[black] (\x.north east) circle (1pt);%
              \fill[black] (\x.north west) circle (1pt);%
              \fill[black] (\x.south east) circle (1pt);%
              \fill[black] (\x.south west) circle (1pt);%
              \path (\x.north west) ++ (135:5pt) coordinate (p1) ;%
              \path (\x.north east) ++ (45:5pt) coordinate (p2) ;%
              \path (\x.south east) ++ (-45:5pt) coordinate (p3) ;%
              \path (\x.south west) ++ (-135:5pt) coordinate (p4) ;%
              \draw[box] (p1) -- (p2) -- (p3) -- (p4) -- cycle;%
            }%
            %% draw lines
            \foreach \x/\y in
            {root/c1,c1/c2,c2/c3,c3/c4,c4/c5,c21/c51,c12/c42,c42/c52,
              c23/c33,c14/c54} {%
              \draw[pointers] (\x.north east) to (\y.north west); %
              \draw[pointers] (\y.south west) to (\x.south east);%
            }%
            %% draw columns
            \foreach \x/\y in
            {c1/c12,c12/c14,c2/c21,c21/c23,c3/c33,c33/c35,c4/c42,
              c5/c51,c51/c52,c52/c54} {%
              \draw[pointers] (\x.south east) to (\y.north east); %
              \draw[pointers] (\y.north west) to (\x.south west);%
            }%
            %% the rest manually
            %% horizontally
            \draw[pointers,shorten >=0pt] (root.south west) to ++(-1/2,0);%
            \draw[pointers,shorten >=0pt] (c12.south west) to ++(-2.5,0);%
            \draw[pointers,shorten >=0pt] (c14.south west) to ++(-2.5,0);%
            \draw[pointers,shorten >=0pt] (c21.south west) to ++(-4.5,0);%
            \draw[pointers,shorten >=0pt] (c23.south west) to ++(-4.5,0);%
            \draw[pointers,shorten >=0pt] (c35.south west) to ++(-6.5,0);%
            % 
            \draw[pointers] (root.north west) ++(-1/2,0) to
            (root.north west);%
            \draw[pointers] (c12.north west) ++(-2.5,0) to
            (c12.north west);%
            \draw[pointers] (c14.north west) ++(-2.5,0) to
            (c14.north west);%
            \draw[pointers] (c21.north west) ++(-4.5,0) to
            (c21.north west);%
            \draw[pointers] (c23.north west) ++(-4.5,0) to
            (c23.north west);%
            \draw[pointers] (c35.north west) ++(-6.5,0) to
            (c35.north west);%
            % 
            \draw[pointers,shorten >=0pt] (c5.north east) to ++(1/2,0);%
            \draw[pointers,shorten >=0pt] (c51.north east) to ++(.5,0);%
            \draw[pointers,shorten >=0pt] (c52.north east) to ++(.5,0);%
            \draw[pointers,shorten >=0pt] (c54.north east) to ++(.5,0);%
            \draw[pointers,shorten >=0pt] (c33.north east) to ++(4.5,0);%
            \draw[pointers,shorten >=0pt] (c35.north east) to ++(4.5,0);%
            % 
            \draw[pointers] (c5.south east) ++(1/2,0) to (c5.south east);%
            \draw[pointers] (c51.south east) ++(.5,0) to (c51.south east);%
            \draw[pointers] (c52.south east) ++(.5,0) to (c52.south east);%
            \draw[pointers] (c54.south east) ++(.5,0) to (c54.south east);%
            \draw[pointers] (c33.south east) ++(4.5,0) to (c33.south east);%
            \draw[pointers] (c35.south east) ++(4.5,0) to (c35.south east);%
            %% 
            %% vertically
            \foreach \x in {c1,c2,c3,c4,c5} {%
              \draw[pointers,shorten >=0pt] (\x.north west) to ++(0,.5);%
              \draw[pointers] (\x.north east) ++(0,.5) to (\x.north east);%
            }%
            \draw[pointers,shorten >=0pt] (c14.south east) to ++(0,-2.5);%
            \draw[pointers,shorten >=0pt] (c54.south east) to ++(0,-2.5);%
            \draw[pointers,shorten >=0pt] (c23.south east) to ++(0,-4.5);%
            \draw[pointers,shorten >=0pt] (c35.south east) to ++(0,-.5);%
            \draw[pointers,shorten >=0pt] (c42.south east) to ++(0,-6.5);%
            
            \draw[pointers] (c14.south west) ++(0,-2.5) to (c14.south west);%
            \draw[pointers] (c54.south west) ++(0,-2.5) to (c54.south west);%
            \draw[pointers] (c23.south west) ++(0,-4.5) to (c23.south west);%
            \draw[pointers] (c35.south west) ++(0,-.5) to (c35.south west);%
            \draw[pointers] (c42.south west) ++(0,-6.5) to (c42.south west);%
          \end{tikzpicture}%
        }%
      \end{minipage}
    };%
    \node [frame,anchor=east,xshift=-1em] (right) at (current
    page.east) {%
      \begin{minipage}{.5\linewidth}
\begin{lstlisting}
typedef struct node_s node_t;
struct node_s {
  node_t *left, *right,
         *up, *down;
  /* etc. */
};
\end{lstlisting}
\begin{lstlisting}
typedef struct column_s column_t;
struct column_s {
  node_t head;
  column_t *left, *right;
  /* etc. */
};
\end{lstlisting}
      \end{minipage}
    };%
  \end{tikzpicture}
\end{frame}

\end{document}
