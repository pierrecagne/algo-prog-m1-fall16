% !TEX program = lualatex
\documentclass[10pt,french,a4paper]{article}

\usepackage{amsmath,amsthm,amssymb}
\usepackage{fontspec}
\usepackage{unicode-math}
\defaultfontfeatures{Ligatures=TeX}
\setmainfont{Cambria}
\setsansfont{Calibri}
%\setmathfont{Cambria Math}
\setmonofont{Consolas}[Scale=MatchLowercase%
]

\usepackage{stmaryrd}
\usepackage{polyglossia}
\usepackage{enumitem}
\usepackage{tikz,tikz-cd}
\usepackage[explicit]{titlesec}
\usepackage{listings}
\usepackage{hyperref}
\usepackage{fmtcount}
% \usepackage{microtype} % uncomment for better rendering

\setmainlanguage{french}

\title{\sffamily TP1: premiers pas en {\tt C}}
\date{20 septembre 2016}
\author{Algorithmique et projet de programmation --- M1 mathématiques} 

\lstdefinestyle{sh}{%
  language=sh,%
  basicstyle=\tt\small,%
}
\lstdefinestyle{C}{%
  language=C,%
  breaklines=true,%
  frame=l,%
  xleftmargin=\parindent,%
  basicstyle=\ttfamily\small,%
  keywordstyle=\bfseries\color{green!40!black!80},%
  showstringspaces=false,%
  commentstyle=\itshape\color{purple!70},%
  identifierstyle=\color{blue!80},%
  stringstyle=\color{red!80},%
  directivestyle=\color{orange!90!black!80},%
  % otherkeywords={},%
  escapeinside={<latex>}{</latex>},%
}
\lstset{style=C}

\theoremstyle{definition}
\newtheorem{exercise}{Exercice}
\theoremstyle{remark}
\newtheorem*{remark}{Remarque}

\newcommand{\shell}[1]{\lstinline[style=sh]|#1|}
\newcommand{\inlinec}[1]{\lstinline[style=C]|#1|}

\begin{document}
\maketitle

\section{Intéraction avec l'utilisateur}
\label{sec:user-io}

Rappelons que l'on utilise la fonction \inlinec{printf}, de la
bibliothèque \inlinec{stdio.h}, pour afficher des chaînes de caractères
à l'écran. Le nom de cette fonction signifie {\em print formatted}, ce
qui signifie qu'elle attend une chaîne de cractère formattée: en
termes plus simples, son argument est une châine de caractères pouvant
contenir des ``jokers''. Par exemple:
\begin{lstlisting}
  printf("Hello, World!\n");

  int n = 10;
  while(n>0){
    printf("%d...", n);
    n--;
  }
  printf("Go!");
\end{lstlisting}

La fonction {\tt C} qui demande une entrée clavier à l'utilisateur est
\inlinec{scanf} (pour {\em scan formatted}), située dans la bibliothèque \inlinec{stdio.h}. On se
contentera de l'utiliser pour demander des {\em nombres} à
l'utilisateurs, comme suit:
\begin{lstlisting}
  int n ;
  scanf("Choose a number: %d", &n); //remark the &
\end{lstlisting}
Remarquez par rapport à l'utilisation de la fonction \inlinec{printf},
l'argument est préfixé par une esperluette ``\inlinec{&}'' devant le
nom de variable (ici \inlinec{n}). C'est qu'il faut passer en
paramètres à \inlinec{scanf} l'{\em adresse} en mémoire de la variable
\inlinec n et non la variable elle-même: on dit que \inlinec{&n} est
une référence sur \inlinec n. Vous comprendrez l'exacte signification
de tout cela quand vous aborderez les pointeurs; en attendant vous
pouvez juste prétendre que c'est une curiosité de la fonction
\inlinec{scanf}.

\begin{exercise}
  Ecrire un programme qui demande deux entiers \inlinec{n} et
  \inlinec{m} à l'utilisateur et lui renvoie le quotient et le reste
  de la division euclidienne de \inlinec{n} par \inlinec{m}
\end{exercise}

\begin{remark}
  La manipulation d'entiers en {\tt C} ne se limite pas au type
  \inlinec{int}. Ci-dessous un tableau récapitulatif des différents
  types, leurs domaines et leurs jokers dans les chaînes formatées:
  \begin{center}
    \begin{tabular}{|c|c|c|c|}
      \hline Nom & Taille & Entiers représentés & Utilisation avec {\tt printf} \\
      \hline {\inlinec{short}} & 16 bits & de $-2^{15}$ à $2^{15}-1$ & {\inlinec{%hd}} \\
      \hline {\inlinec{unsigned short}} & 16 bits & de $0$ à $2^{16}-1$ & {\inlinec{%hu}} \\
      \hline {\inlinec{int}} & 32 bits & de $-2^{31}$ à $2^{31}-1$ & {\inlinec{%d}} \\
      \hline {\inlinec{unsigned int}} & 32 bits & de $0$ à $2^{32}-1$ & {\inlinec{%u}} \\
      \hline {\inlinec{long}} & 64 bits & de $-2^{63}$ à $2^{63}-1$ & {\inlinec{%l}} \\
      \hline {\inlinec{unsigned long}} & 64 bits & de $0$ à $2^{64}-1$ & {\inlinec{%lu}} \\
      \hline
    \end{tabular}
  \end{center}
\end{remark}

\end{document}
