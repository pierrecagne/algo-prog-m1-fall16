% !TEX program = lualatex
\documentclass[french,a4paper]{article}

\usepackage{amsmath,amsthm,amssymb}
\usepackage{fontspec}
\usepackage{unicode-math}
\defaultfontfeatures{Ligatures=TeX}
% \setmainfont{Cambria}
% \setsansfont{Calibri}
% %\setmathfont{Cambria Math}
% \setmonofont{Consolas}[Scale=MatchLowercase]

\usepackage{stmaryrd}
\usepackage{polyglossia}
\usepackage{enumitem}
\usepackage{tikz,tikz-cd}
\usepackage[explicit]{titlesec}
\usepackage{listings}
\usepackage{hyperref}
\usepackage{fmtcount}
\usepackage{microtype} % uncomment for better rendering

\setmainlanguage{french}

\title{\sffamily TP1: premiers pas en {\tt C}}
\date{20 septembre 2016}
\author{Algorithmique et projet de programmation --- M1 mathématiques} 

\lstdefinestyle{sh}{%
  language=sh,%
  basicstyle=\tt\small,%
}
\lstdefinestyle{C}{%
  language=C,%
  breaklines=true,%
  frame=l,%
  xleftmargin=\parindent,%
  basicstyle=\ttfamily\small,%
  keywordstyle=\bfseries\color{green!40!black!80},%
  showstringspaces=false,%
  commentstyle=\itshape\color{purple!70},%
  identifierstyle=\color{blue!80},%
  stringstyle=\color{red!80},%
  directivestyle=\color{orange!90!black!80},%
  % otherkeywords={},%
  escapeinside={<latex>}{</latex>},%
}
\lstset{style=C}

\theoremstyle{definition}
\newtheorem{exercise}{Exercice}
\theoremstyle{remark}
\newtheorem*{remark}{Remarque}

\newcommand{\shell}[1]{\lstinline[style=sh]|#1|}
\newcommand{\inlinec}[1]{\lstinline[style=C]°#1°}

\begin{document}
\maketitle

\section{Intéraction avec l'utilisateur}
\label{sec:user-io}

Rappelons que l'on utilise la fonction \inlinec{printf}, de la
bibliothèque \inlinec{stdio.h}, pour afficher des chaînes de caractères
à l'écran. Le nom de cette fonction signifie {\em print formatted}, ce
qui signifie qu'elle attend une chaîne de cractère formattée: en
termes plus simples, son argument est une châine de caractères pouvant
contenir des ``jokers''. Par exemple:
\begin{lstlisting}
  printf("Hello, World!\n");

  int n = 10;
  while(n>0){
    printf("%d...", n);
    n--;
  }
  printf("Go!");
\end{lstlisting}

La fonction {\tt C} qui demande une entrée clavier à l'utilisateur est
\inlinec{scanf} (pour {\em scan formatted}), située dans la bibliothèque \inlinec{stdio.h}. On se
contentera de l'utiliser pour demander des {\em nombres} à
l'utilisateurs, comme suit:
\begin{lstlisting}
  int n ;
  printf("Choose a number: ");
  scanf("%d", &n); //remark the &
\end{lstlisting}
Remarquez, par rapport à l'utilisation de la fonction
\inlinec{printf}, que l'argument est préfixé par une esperluette
``\inlinec{&}'', ajoutée {\bf juste devant} le nom de variable (ici
\inlinec{n}). C'est qu'il faut passer en paramètres à \inlinec{scanf}
l'{\em adresse} en mémoire de la variable \inlinec n et non la
variable elle-même: on dit que \inlinec{&n} est une référence sur
\inlinec n. Vous comprendrez l'exacte signification de tout cela quand
vous aborderez les pointeurs; en attendant vous pouvez juste prétendre
que c'est une curiosité de la fonction \inlinec{scanf}.

\begin{exercise}
  \'Ecrire un programme qui demande deux entiers \inlinec{n} et
  \inlinec{m} à l'utilisateur et lui renvoie le quotient et le reste
  de la division euclidienne de \inlinec{n} par \inlinec{m}.
\end{exercise}

\section{Conditions booléennes}
\label{sec:bool}

Les structures de contrôle {\tt\small if-then-else} permettent
d'influencer le comportement du programme en fonction de la valeur de
vérité d'un booléen. Typiquement, ces valeurs booléennes seront de
comparaisons numériques. Exemple:
\begin{lstlisting}
  if(n!=0){
    printf("Division by %n allowed.\n", n);
  }
  else {
    printf("Division by 0 is not allowed.\n");
  }
\end{lstlisting}

Le tableau ci-dessous donne les relations de comparaison numérique en
{\tt C}:
\begin{center}
  \begin{tabular}{|c|c|c|}
    \hline Relation & Nom & Comportement attendu \\
    \hline {\inlinec{a<b}} & lesser than & {vrai ssi \inlinec{a} strictement inférieur à \inlinec{b}} \\
    \hline {\inlinec{a<=b}} & lesser than & {vrai ssi \inlinec{a} inférieur ou égal à \inlinec{b}} \\
    \hline {\inlinec{a>b}} & lesser than & {vrai ssi \inlinec{a} strictement supérieur à \inlinec{b}} \\
    \hline {\inlinec{a>=b}} & lesser than & {vrai ssi \inlinec{a} supérieur ou égal à \inlinec{b}} \\
    \hline {\inlinec{a==b}} & lesser than & {vrai ssi \inlinec{a} égal à \inlinec{b}} \\
    \hline {\inlinec{a!=b}} & lesser than & {vrai ssi \inlinec{a} différent de \inlinec{b}} \\
    \hline
  \end{tabular}
\end{center}

\noindent On peut également combiner des booléens pour obtenir des
conditions plus complexes. Les opérations (les plus courantes) sur les
booléens en {\tt C} sont les suivantes:
\begin{center}
  \begin{tabular}{|c|c|c|}
    \hline Opération & Nom & Comportement attendu \\
    \hline {\inlinec{bool1 && bool2}} & conjunction & {vrai ssi \inlinec{bool1} et \inlinec{bool2} sont vrais} \\
    \hline {\inlinec{bool1 || bool2}} & disjunction & {vrai ssi l'un de \inlinec{bool1} et \inlinec{bool2} est vrai} \\
    \hline {\inlinec{!bool}} & negation & {vrai ssi \inlinec{bool} est faux} \\
    \hline
  \end{tabular}
\end{center}

\begin{exercise}
  \'Ecrire un programme qui demande à l'utilisateur d'entrer une année
  et détermine si celle-ci est bissextile. Rappelons qu'une année est
  bissextile si elle est divisible par 4 mais pas par 100, ou alors
  par 400.
\end{exercise}

\section{C'est l'histoire d'un type...}
\label{sec:type-integers}

\begin{remark}
  La manipulation d'entiers en {\tt C} ne se limite pas au type
  \inlinec{int}. Ci-dessous un tableau récapitulatif des différents
  types, leurs domaines et leurs jokers dans les chaînes formatées:
  \begin{center}
    \begin{tabular}{|c|c|c|c|}
      \hline Nom & Taille & Entiers représentés & Utilisation avec {\inlinec{printf}} \\
      \hline {\inlinec{short}} & 16 bits & de $-2^{15}$ à $2^{15}-1$ & {\inlinec{\%hd}} \\
      \hline {\inlinec{unsigned short}} & 16 bits & de $0$ à $2^{16}-1$ & {\inlinec{\%hu}} \\
      \hline {\inlinec{int}} & 32 bits & de $-2^{31}$ à $2^{31}-1$ & {\inlinec{\%d}} \\
      \hline {\inlinec{unsigned int}} & 32 bits & de $0$ à $2^{32}-1$ & {\inlinec{\%u}} \\
      \hline {\inlinec{long}} & 64 bits & de $-2^{63}$ à $2^{63}-1$ & {\inlinec{\%l}} \\
      \hline {\inlinec{unsigned long}} & 64 bits & de $0$ à $2^{64}-1$ & {\inlinec{\%lu}} \\
      \hline
    \end{tabular}
  \end{center}
\end{remark}

\begin{exercise}
  La suite de Fibonacci est définie par $u_0=1$, $u_1=1$ et
  $u_n = u_{n-1} + u_{n-2}$ pour tout $n\geq 2$. \'Ecrire un programme
  qui calcule toutes les valeurs possibles de la suite de Fibonacci et
  s'arrête quand le résultat dépasse le domaine du type choisi pour
  représenter les entiers dans votre programme.
\end{exercise}

\begin{exercise}
  Modifier le programme précédent pour laisser le choix à
  l'utilisateur du type à utiliser. Indice : on affichera en début de
  programme un menu de la forme
  \begin{lstlisting}[language=sh,identifierstyle=\color{black}]
  1. short
  2. unsigned short
  3. int
  4. unsigned int
  5. long
  6. unsigned long
  Type to use (enter corresponding number):
  \end{lstlisting}
\end{exercise}

\section{Let's play}
\label{sec:games}

La bibliothèque \inlinec{stdlib.h} contient une fonction qui génère un
nombre aléatoire: \inlinec{rand()} renvoie un entier aléatoire entre
\inlinec{0} et une (grande) valeur abitraire fixée par {\tt C} et
appelée \inlinec{RAND_MAX}. Pour que \inlinec{rand} se comporte
réellement comme un génrateur aléatoire, il faut l'initialiser avec
une {\em graine}, par exemple l'heure: \lstinputlisting{random.c}

\begin{exercise}
  \'Ecrire un programme qui génère un nombre aléatoire entre 0 et un
  entier \inlinec{n} fixé (exclu).
\end{exercise}

\begin{exercise}
  \'Ecrire un jeu du {\em plus ou moins}. Le jeu se déroule comme suit:
  \begin{enumerate}[label=(\arabic*)]
  \item un nombre \inlinec{bound} est fixé à l'avance,
  \item un nombre de coups \inlinec{chances} est fixé à l'avance,
  \item le programme génère un nombre aléatoirement entre \inlinec{0}
    et \inlinec{bound},
  \item l'utilisateur est invité à entrer un nombre et le programme
    indique si le nombre cherché est plus grand ou plus petit (ou
    égal),
  \item on répète cette instruction jusqu'à ce que le joueur trouve le
    nombre (et dans ce cas il gagne) ou que le nombre de coups soit
    écoulé (et dans ce cas il perd).
  \end{enumerate}
\end{exercise}

\begin{exercise}
  Améliorer le programme en proposant différent niveau à
  l'utilisateur. On pourra par exemple afficher un menu en début de
  programme de la forme:
  \begin{lstlisting}[language=sh,identifierstyle=\color{black}]
  1. Level 1 -- number between 0 and 10 -- 5 chances
  2. Level 2 -- number between 0 and 100 -- 20 chances
  3. Level 3 -- number between 0 and 100 -- 5 chances
  4. Level 4 -- number between 0 and 1000 -- 10 chances
  Level (enter corresponding number):
  \end{lstlisting}
\end{exercise}

\begin{exercise}
  Améliorer encore un peu le jeu en ajoutant une dernière option au
  menu où on laisse le joueur libre de rentrer lui même les nombres
  \inlinec{bound} et \inlinec{chances}.
\end{exercise}

\section{\`A rendre pour le 27 septembre 2016}
\label{sec:homeworks}

\begin{exercise}
  \'Ecrire un programme demandant à l'utilisateur de rentrer un entier
  et retournant si celui-ci est premier ou non. Dans le cas où le
  nombre n'est pas premier, on affichera le plus petit diviseur
  (strictement supérieur à 1).
\end{exercise}

\begin{exercise}
  Compiler sans le flag \shell{-Wall} et exécuter le programme
  suivant: \lstinputlisting{weird.c} Le corriger pour qu'il ait le
  comportement attendu et expliquer (en commentaire de votre code) ce
  qu'il se passe dans le programme original. Indice: recompiler avec
  \shell{-Wall}.
\end{exercise}

\end{document}
