% !TEX program = lualatex
\documentclass[french,a4paper]{article}

\usepackage{amsmath,amsthm,amssymb}
\usepackage{fontspec}
% \usepackage{unicode-math}
%\defaultfontfeatures{Ligatures=TeX}
\setmainfont[Ligatures=TeX]{Cambria}
\setsansfont[Ligatures=TeX]{Calibri}
%\setmathfont{Cambria Math}
\setmonofont{Consolas}[Scale=MatchLowercase]

\usepackage{stmaryrd}
\usepackage{polyglossia}
\usepackage{enumitem}
\usepackage{tikz,tikz-cd}
\usepackage[explicit]{titlesec}
\usepackage{listings}
\usepackage{hyperref}
\usepackage{fmtcount}
\usepackage{microtype} % uncomment for better rendering

\setmainlanguage{french}

\title{\sffamily TP4: tableaux, pointeurs}%
\date{27 septembre 2016}%
\author{Algorithmique et projet de programmation --- M1
  mathématiques}%

\lstdefinestyle{sh}{%
  language=sh,%
  basicstyle=\tt\small,%
}
\lstdefinestyle{C}{%
  language=C,%
  breaklines=true,%
  frame=l,%
  xleftmargin=\parindent,%
  basicstyle=\ttfamily\small,%
  keywordstyle=\bfseries\color{green!40!black!80},%
  showstringspaces=false,%
  commentstyle=\itshape\color{purple!70},%
  identifierstyle=\color{blue!80},%
  stringstyle=\color{red!80},%
  directivestyle=\color{orange!90!black!80},%
  upquote=true,%
  otherkeywords={fun_t,arg1_t,arg2_t,typ},%
  escapeinside={<latex>}{</latex>},%
}
\lstset{style=C}

\theoremstyle{definition}
\newtheorem{exercise}{Exercice}
\theoremstyle{remark}
\newtheorem*{remark}{Remarque}

\newcommand{\shell}[1]{\lstinline[style={},style=sh]|#1|}
\newcommand{\inlinec}[1]{\lstinline[style=C]°#1°}

\begin{document}
\maketitle

\begin{exercise}[Warm up]
  Créer les fonctions suivantes traitant des tableaux d'entiers:
  \lstinputlisting{warmup.h}
\end{exercise}

\section{Tris}
\label{sec:sort}

Commençons par les tris naïfs, qui s'éxécute en
$\operatorname{O}(n^2)$.

\begin{exercise}[Par sélection]
  Le tri par sélection opère de la façon suivante: il parcourt le
  tableau à la recherche du plus grand élément, puis l'échange avec le
  dernier élement du tableau ; il cherche ensuite le plus grand
  élément parmi les valeurs restantes et l'échange avec l'avant
  dernier élément du tableau ; etc.

  \'Ecrire une fonction qui implémente le tri par sélection:
  \begin{lstlisting}
void select_sort (int size, int *array);
  \end{lstlisting}
  Indice: en {\tt C}, un tableau de taille $n$ est tout autant un
  tableau de taille $i<n$.
\end{exercise}

\begin{exercise}[\`A bulles]
  Le tri à bulles repose sur le même principe que le tri par
  sélection: placer l'élément maximal en dernière position et
  recommencer sur le sous-tableau des éléments restants. En revanche,
  on ne cherche pas le maximum explicitement: on va le «~faire
  remonter~» (d'où le nom de {\em tri à bulles}) en inversant
  localement les valeurs des cases $i$ et $i+1$ quand la première est
  supérieure à la seconde.

  \'Ecrire une fonction qui implémente le tri à bulles:
  \begin{lstlisting}
void bubble_sort (int size, int *array);    
  \end{lstlisting}
\end{exercise}

\begin{exercise}
  Le tri par insertion change le point de vue: considérant que la fin
  d'un tableau, des indices $i$ à $n-1$, est déjà trié, il suffit
  d'insérer la valeur de la case $i$ au bon endroit pour avoir une fin
  triée d'une case de plus. En itérant le processus depuis une fin
  vide jusqu'à une fin de la taille du tableau, on aura trié le
  tableau en entier.

  \'Ecrire une fonction qui implémente le tri par insertion:
  \begin{lstlisting}
void insert_sort (int size, int i, int *array);    
  \end{lstlisting}
  qui suppose que le tableau \inlinec{array} de taille \inlinec{size}
  est trié entre les index \inlinec{i+1} et \inlinec{n-1}, et qui trie
  entièrement le tableau \inlinec{array}.
\end{exercise}

On passe maintenant au tri dont la complexité est
$\operatorname{O}(n \log n)$.

\begin{exercise}[Rapide]
  Le tri rapide trie les valeurs entre les cases d'index \inlinec{min}
  et \inlinec{max} d'un tableau de la façon suivante: on choisit une
  valeur $p$ (appelée pivot) du tableau à trier (souvent on prend
  juste la dernière valeur du tableau), puis on met dans n'importe
  quel ordre tous les éléments inférieurs au pivot à sa gauche et tous
  les élements supérieurs à sa droite ; on réitère sur chacun des deux
  sous-tableaux gauche et droit ainsi créés.

  \'Ecrire une fonction qui implémente le tri rapide:
  \begin{lstlisting}
void quick_sort (int min, int max, int *array);
  \end{lstlisting}
\end{exercise}

\begin{exercise}[Fusion]
  Le tri fusion se base sur le principe «~diviser pour régner~»: trier
  un tableau revient à trier la première moitié et la deuxième moitié
  indépendemment, puis à les fusionner. La fusion suppose voir deux
  tableaux triés de même taille,qu'elle parcoure simultanément,
  insérant dans un nouveau tableau le plus petit des deux éléments lus
  à chaque fois.

  \'Ecrire une fonction:
  \begin{lstlisting}
void merge (int start, int mid, int end, int *array);
  \end{lstlisting}
  qui suppose que les sous-tableaux
  \inlinec{\{array[start],...,array[mid-1]\}} et
  \inlinec{\{array[mid],...,array[end]\}} sont triés, et écrit la
  fusion des deux dans le sous-tableaux
  \inlinec{\{array[start],...,array[end]\}}. On n'hésitera pas à
  utiliser un tableau intermédiaire temporaire (la modification {\em
    en place} étant difficile à réaliser).

  \'Ecrire une fonction récursive (en s'aidant bien entendu de la
  fonction précédente) qui implémente le tri fusion pour le
  sous-tableau compris entre \inlinec{start} et \inlinec{end} d'un
  tableau \inlinec{array} :
  \begin{lstlisting}
void merge_sort (int start, int end, int* array);
  \end{lstlisting}
\end{exercise}

\section{Autres exemples}
\label{sec:other}

\begin{exercise}[Crible d'\'Eratosthène]
  Au {\sc III}\textsuperscript{e} scièle avant J-C., \'Eratosthène
  invente une manière astucieuse de générer tous les nombres premiers
  jusqu'à une certaine borne $n$. Il suffit pour cela d'écrire les
  entiers de $2$ à $n$ sur une feuille puis de répéter l'action
  suivante en partant de $2$: si le nombre n'est pas barré, on barre
  tous ces multiples sauf lui-même ; si le nombre est barré, on passe
  au suivant. Lorsqu'on arrive à $n$, les nombres non barrés sont
  exactement ceux qui ne sont multiples de personne à part eux même,
  c'est-à-dire les nombre premiers.

  Implémenter le crible d'\'Eratosthène en une fonction:
  \begin{lstlisting}
void sieve(int n, char *era);
  \end{lstlisting}
  où le tableau \inlinec{era} contiendra \inlinec{0} dans les cases
  indexées par des non-premiers et \inlinec{1} dans celles indexées
  par des premiers.
\end{exercise}

\begin{exercise}[Euclide étendu, le retour]
  Maintenant que l'on connaît les pointeurs, on peut implémenter une
  version plus simple de l'algorithme d'Euclide étendu. Rappelons que
  cet algorithme renvoie, pour tout couple $(a,b) \in \mathbb N$, le
  pgcd de $a$ et $b$, ainsi que deux entiers $u,v \in \mathhbb Z$ tels
  que $ua+bv = \operatorname{gcd}(a,b)$. Le problème est qu'un
  programme {\tt C} n'est autorisé à renvoyer qu'une seule valeur!
  Heureusement, grâce aux pointeurs, on peut contourner ce problème.

  Implémenter l'algorithme d'Euclide comme une fonction récursive de
  signature:
  \begin{lstlisting}
void ext_euclid (int a, int b, int *gcd, int *u, int *v);    
  \end{lstlisting}
\end{exercise}

\section{\`A rendre pour la prochaine fois}
\label{sec:homeworks}

\begin{exercise}~
  \begin{enumerate}[label=(\roman*)]
  \item \'Ecrire une fonction, la plus naïve possible, renvoyant la
    somme maximale que l'on peut obtenir en additionnant des éléments
    consécutifs d'un tableau \inlinec{array} de taille \inlinec{size}:
    \begin{lstlisting}
int max_sum (int size, int *array)
    \end{lstlisting}
    Par exemple, la fonction doit renvoyer \inlinec{11} sur le tableau
    \inlinec{\{-1,1,8,-7,9,-2\}} (réalisé par les éléments consécutifs
    $1,8,-7,9$).
  \item Estimer la complexité de la fonction précédente (en fonction
    de la taille du tableau).
  \item Le but de cette question est d'obtenir un algorithme plus
    efficace. Notons $m_i$ la somme maximale d'éléments consécutifs
    pour le sous-tableau de \inlinec{array} donné par les indexs de
    $0$ à $i$. Par exemple, $m_0$ est $0$ (la somme vide) si
    \inlinec{array[0]} est négatif et vaut \inlinec{array[0]} sinon.

    En exprimant $m_{i+1}$ en fonction de $m_i$, écrire une fonction
    calculant la même chose qu'en question (a) mais plus efficacement:
    \begin{lstlisting}
int max_sum_better (int size, int *array);      
    \end{lstlisting}
  \item Quelle est la complexité de cette nouvelle fonction ?
  \end{enumerate}

  Cet exercice est une initiation à une technique que vous allez
  beaucoup utiliser par la suite: la programmation dynamique.
\end{exercise}

\end{document}