% !TEX program = lualatex
\documentclass[french,a4paper]{article}

\usepackage{amsmath,amsthm,amssymb}
\usepackage{fontspec}
% \usepackage{unicode-math}
%\defaultfontfeatures{Ligatures=TeX}
\setmainfont[Ligatures=TeX]{Cambria}
\setsansfont[Ligatures=TeX]{Calibri}
%\setmathfont{Cambria Math}
\setmonofont{Consolas}[Scale=MatchLowercase]

\usepackage{stmaryrd}
\usepackage{polyglossia}
\usepackage{enumitem}
\usepackage{tikz,tikz-cd}
\usepackage[explicit]{titlesec}
\usepackage{listings}
\usepackage{hyperref}
\usepackage{fmtcount}
\usepackage{microtype} % uncomment for better rendering

\setmainlanguage{french}

\title{\sffamily TP6: manipulation de fichiers, piles et files}%
\date{8 novembre 2016}%
\author{Algorithmique et projet de programmation --- M1
  mathématiques}%

\lstdefinestyle{sh}{%
  language=sh,%
  basicstyle=\tt\small,%
}
\lstdefinestyle{C}{%
  language=C,%
  %breaklines=true,%
  frame=l,%
  xleftmargin=\parindent,%
  basicstyle=\ttfamily\small,%
  keywordstyle=\bfseries\color{green!40!black!80},%
  showstringspaces=false,%
  commentstyle=\itshape\color{purple!70},%
  identifierstyle=\color{blue!80},%
  stringstyle=\color{red!80},%
  directivestyle=\color{orange!90!black!80},%
  upquote=true,%
  otherkeywords={stack_t,queue_t},%
  escapeinside={<latex>}{</latex>},%
}
\lstset{style=C}

\theoremstyle{definition}
\newtheorem{exercise}{Exercice}
\theoremstyle{remark}
\newtheorem*{remark}{Remarque}

\newcommand{\shell}[1]{\lstinline[style={},style=sh]|#1|}
\newcommand{\inlinec}[1]{\lstinline[style=C]°#1°}

\begin{document}
\maketitle

\section{Manipulation de fichiers}
\label{sec:sort-unix}

On va s'atteler à recoder (une version simple de) la commande
\shell{sort} des sytèmes UNIX. C'est un petit programme qui prend en
paramètre un chemin d'accès à un fichier texte et affiche à l'écran
l'ensemble de ses lignes triées alphabétiquement. Ainsi, si un fichier
\shell{phonebook} contient:
\lstinputlisting[style={},style=sh]{phonebook.txt} L'exécution de la
commande \shell{sort phonebook} affichera:
\begin{lstlisting}[style={},style=sh]
Adrien 0694367833
Bernard 0663896735
Claudia 0642356732
Clement 0678900034
Lea 0612673546
Zaid 0673894765
\end{lstlisting}

\begin{exercise}
  \'Ecrire une fonction qui compare deux chaînes de caractères dans
  l'ordre lexicographique:
  \begin{lstlisting}
char compare_str (char* s, char* t);
  \end{lstlisting}
  Elle renverra \inlinec{1} si \inlinec{s} est inférieur à \inlinec{t}
  au sens large et \inlinec{0} sinon. On rappelle que
  \inlinec{s}${}\leq{}$\inlinec{t} si et seulement si
  \begin{enumerate}[label=-]
  \item \inlinec{s} est un préfixe de \inlinec{t}
  \item il existe $i\geq 0$ tel que \inlinec{s[k] = t[k]} pour tout
    $0\leq k\leq i-1$ et \inlinec{s[i]}${}\leq{}$\inlinec{t[i]}.
\end{enumerate}

\end{exercise}

\begin{exercise}
  \'Ecrire une fonction adaptée des fonctions de tri vues la dernière fois:
  \begin{lstlisting}
void sort_str (char** strs);
  \end{lstlisting}
  qui trie alphabétiquement un tableau de chaînes de caractères.
\end{exercise}

Rappelons que pour traiter les fichiers textes, le langage {\tt C}
dispose des fonction suivantes, disponibles dans la bibliothèque
\inlinec{stdio.h}: \lstinputlisting{fopen.c}

\begin{exercise}
  En déduire un programme implémentant la commande \shell{sort}
  décrite précédemment. En particulier, on déterminera le nombre de
  lignes et la taille maximale des lignes du fichier passé en
  paramètre pour allouer un tableau de taille suffisante.
\end{exercise}

\section{Piles et files d'attente}
\label{sec:stack}

La structure de pile ({\em stack} en anglais) est très utile en
informatique. Notamment, elle modélise bien la forme de la mémoire et
est donc particulièrement adaptée à l'écriture d'interpréteur (cf.\
section \ref{sec:homeworks}). Rappelons qu'une pile est une structure
de données modélisant un ensemble fini et munie de deux fonctions
dédiées: \inlinec{stack_push} qui va ajouter un élément à la pile (on
dit parfois {\em empiler}), et \inlinec{stack_pop} qui supprime et
retourne le dernier élément ajouté à la pile (on dit parfois {\em
  dépiler}).

Il existe de nombreuses manières d'implémenter les piles. Néanmoins,
la plupart du temps, on utilise une liste chaînée:
\inlinec{stack_push} ajoute l'élément en tête de liste, et
\inlinec{stack_pop} retire la tête de liste. Ces deux actions se font
donc en temps constant\footnote{Et c'est ça le plus important: quand
  on manipule des piles, on est amener à empiler et dépiler un très
  grand nombre de fois.}.

\begin{exercise}
  \'Ecrire une structure modélisant les piles de chaînes de
  caractères. (On peut évidemment retourner chercher du code des
  séances précédentes...)
\end{exercise}

\begin{exercise}
  \'Ecrire les fonctions d'empilement et dépilement:
  \begin{lstlisting}
stack_t stack_push (char *string, stack_t s);
char *stack_pop (stack_t s); // here 's' gets modified
  \end{lstlisting}
\end{exercise}

\medskip

Les files d'attente ({\em queue} en anglais) jouent un rôle dual à
celui des piles: elles modélisent tout autant des ensembles finis, et
viennent également avec deux fonctions dédiées \inlinec{queue_push} et
\inlinec{queue_pop}. Mais si \inlinec{queue_push} ajoute bien un
élément à la file d'attente, la fonction \inlinec{queue_pop} supprime
et retourne l'élément le plus anciennement ajouté à la
file. C'est-à-dire que si on ajoute à une file vide, d'abord $1$ puis
$2$ puis $3$, la fonction \inlinec{queue_pop} doit retourner $1$
(alors que la même opération sur une pile aurait retourné $3$).

La structure classique implémentant les files d'attente sont les
listes {\em doublement} chaînées. Cette structure est très proche des
listes chaînées mais chaque cellule comporte deux pointeurs: un vers
la cellule suivante (comme pour les listes chaînées) et un vers la
cellule précédente. Remarquons alors qu'il n'y a plus de {\em sens}
dans une liste doublement chaînée (qui est la tête?). Ainsi, au même
titre qu'on identifiait une liste chaînée avec un pointeur sur la
première cellule, on identifie une liste doublement chaînée avec un
couple de pointeurs, un pour chaque extrêmité de la liste. La fonction
\inlinec{queue_push} ajoute alors un élément à une extrêmité, tandis
que \inlinec{queue_pop} retire un élément à l'autre.

\begin{exercise}
  \'Ecrire une structure modélisant les files d'attente de
  chaînes de caractères. Indice: cela aura la forme suivante
  \begin{lstlisting}
typedef struct cell_s cell_t;
typedef struct queue_s queue_t;

struct cell_s {
  /* ... inspired from cells of linked list ... */
};
struct queue_s {
  cell_t* first;
  cell_t* last;
};
  \end{lstlisting}
\end{exercise}

\begin{exercise}
  \'Ecrire les fonctions:
  \begin{lstlisting}
queue_t queue_push (char *string, queue_t q);
char *queue_pop (queue_t q); // here 'q' gets modified
  \end{lstlisting}
\end{exercise}

\section{Pour le 15 novembre 2016}
\label{sec:homeworks}

\`A partir de maintenant, le projet doit être la priorité. Aussi les
exercices à rendre d'une séance sur l'autre deviennent-t-ils
facultatifs. Néanmoins, j'invite quiconque ayant le temps de les faire
et de les envoyer par mail comme auparavant.

\begin{exercise}[Calculatrice]
  %
  Le but de cet exercice est d'implémenter une calculatrice en
  utilisant la syntaxe dite {\em polonaise}. Dans cette syntaxe, on
  marque les opérations $+,*,-,/$ en préfixe de ces arguments: par
  exemple, on écrit \shell{+ 1 2} pour $1+2$. Cela a l'avantage de ne
  pas avoir à parenthèser son calcul. En effet, en notation habituelle
  (on dit notation {\em infixe}), le calcul $1+2*3$ peut-être
  interprété comme: $(1+2)*3$ ou $1+(2*3)$; la notation polonaise les
  distingue d'emblée: le premier étant \shell{* + 1 2 3} et le
  deuxième \shell{+ 1 * 2 3}.
  %
  \begin{enumerate}
  \item \'Ecrire une fonction
    \begin{lstlisting}
stack_t str_to_stack (char *s);
    \end{lstlisting}
    qui prend en paramètre une chaîne de caractères contenant une
    suite d'opérations en notation polonaise et qui retourne une pile
    de chaînes de caractères contenant l'empilement successif des
    chaînes de caractères rencontrées (en considérant qu'un espace
    sépare deux telles chaîens). Par exemple, dépiler et afficher tous
    les éléments de \inlinec{str_to_stack("+ 1 * 2 3")} doit donner
    \shell{ 3 2 * 1 +}.
  \item \'Ecrire une fonction récursive
    \begin{lstlisting}
int compute (stack_t s);
    \end{lstlisting}
    qui maintient une deuxième pile \inlinec{t} de la manière suivante
    tant que \inlinec{s} n'est pas vide:
    \begin{itemize}
    \item on dépile \inlinec{s},
    \item soit l'élément dépilé est une chaîne de caractères contenant
      un entier, auquel cas on l'empile sur \inlinec{t},
    \item soit l'élément dépilé est un des 4 caractères suivants:
      \inlinec{"+"},\inlinec{"*"},\inlinec{"-"},\inlinec{"/"}; on
      dépile alors deux fois \inlinec{t} et on effectue l'opération
      avec ces deux éléments dépilés, avant d'ampiler le résultat
      (sous forme de chaîne de caractères) sur \inlinec{t}.
    \end{itemize}
    Quand la pile \inlinec{s} est vide, \inlinec{t} doit contenir une
    unique case contenant le résultat (sous forme de chaîne de
    caractères).
  \item Coder une calculatrice complète, demandant à l'utilisateur de
    rentrer un calcul en notation polonaise et retournant le résultat
    de ce calcul.
  \end{enumerate}
\end{exercise}

\begin{exercise}[Plus difficile]
  Dans l'exercice précédent, la fonction \inlinec{compute} est obligée
  de manipuler des châines de caratères alors qu'on sait bien qu'on
  manipule en fait des entiers stockés dans de telles chaînes et des
  opérations. On peut éviter cela, à condition que les cellules de
  notre pile soit un peu plus élaborées.

  En utilisant le mot clé \inlinec{union}, créer un type qui peut
  contenir alternativement des entiers et des caractères. Implémenter
  les piles pour ce nouveau type. Modifier le programme précédent pour
  qu'il manipule des piles de ce type.
\end{exercise}

\end{document}