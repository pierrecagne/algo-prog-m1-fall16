% !TEX program = lualatex
\documentclass[french,a4paper]{article}

\usepackage{amsmath,amsthm,amssymb}
\usepackage{fontspec}
% \usepackage{unicode-math}
%\defaultfontfeatures{Ligatures=TeX}
\setmainfont[Ligatures=TeX]{Cambria}
\setsansfont[Ligatures=TeX]{Calibri}
%\setmathfont{Cambria Math}
\setmonofont{Consolas}[Scale=MatchLowercase]

\usepackage{stmaryrd}
\usepackage{polyglossia}
\usepackage{enumitem}
\usepackage{tikz,tikz-cd}
\usepackage[explicit]{titlesec}
\usepackage{listings}
\usepackage{hyperref}
\usepackage{fmtcount}
\usepackage{microtype} % uncomment for better rendering

\setmainlanguage{french}

\title{\sffamily TP8: arbres binaires de recherche et programmation
 dynamique}%
\date{22 novembre 2016}%
\author{Algorithmique et projet de programmation --- M1
  mathématiques}%

\lstdefinestyle{sh}{%
  language=sh,%
  basicstyle=\tt\small,%
}
\lstdefinestyle{C}{%
  language=C,%
  %breaklines=true,%
  frame=l,%
  xleftmargin=\parindent,%
  basicstyle=\ttfamily\small,%
  keywordstyle=\bfseries\color{green!40!black!80},%
  showstringspaces=false,%
  commentstyle=\itshape\color{purple!70},%
  identifierstyle=\color{blue!80},%
  stringstyle=\color{red!80},%
  directivestyle=\color{orange!90!black!80},%
  upquote=true,%
  otherkeywords={stack_t,queue_t},%
  escapeinside={<latex>}{</latex>},%
}
\lstset{style=C}

\theoremstyle{definition}
\newtheorem{exercise}{Exercice}
\theoremstyle{remark}
\newtheorem*{remark}{Remarque}

\newcommand{\shell}[1]{\lstinline[style={},style=sh]|#1|}
\newcommand{\inlinec}[1]{\lstinline[style=C]°#1°}

\begin{document}
\maketitle

\section{Arbres binaires de recherche}
\label{sec:bst}

On va coder une petite {\em bibliothèque} implémentant les arbres
binaires de recherche vus en cours ({\em Binary Search Trees} en
anglais). Rappelons qu'un arbre binaire de recherche est un arbre
binaire où chaque noeud contient un élément strictement supérieur aux
noeuds de son sous-arbre gauche et strictement inférieur aux noeuds de
son sous-arbre droit (on a aussi les variantes non strictes d'un des
deux côtés). On peut y faire une recherche d'élément, une suppression
ou un ajout en $O(h)$ avec $h$ la hauteur de l'arbre. En particulier,
si l'arbre est équilibré, alors $h \sim \ln n$ avec $n$ le nombre de
noeuds et toutes ces opérations sont donc en $O(\ln n)$. Cette
structure peut être vue comme intermédiaire entre la structure de
tableau (accès en $O(1)$, mais ajout et suppression en $O(n)$) et
celle de liste chaînées (ajout en $O(1)$, mais accès et suppression en
$O(n)$). De ce fait, on peut s'en servir pour représenter les
ensembles finis de manière assez efficace.

\medskip

On va donc créer une bibliothèque qui fait cela: en particulier,
l'utilisateur n'aura pas à connaître l'implémentation des arbres
binaires de recherche. Tout ce à quoi il aura accès sera une
collection de fonction permettant la manipulation d'ensembles finis
d'entiers. Commençons donc par créer un fichier \shell{bst.c} qui
contiendra toutes les subtilités de l'implémentation que l'on veut
cacher à l'utilisateur. Il devra tout d'abord contenir la structure
représentant un noeud de l'arbre binaire:
\begin{lstlisting}
struct bst_node_s {
  int key;
  struct bst_node_s *left, *right;
}
\end{lstlisting}
et tout un tas de fonction que l'on va coder par la suite.

L'utilisateur de notre bibliothèque n'a pas besoin de connaître cette
structure: tout ce dont il a besoin est de savoir qu'il y a un type
représentant les ensembles finis d'entiers, et que celui-ci prend la
forme d'un pointeur vers une structure (obscure pour
lui\footnote{C'était exactement notre cas lors de l'utilisation du
  type \inlinec{FILE*} dans la manipualtion de fichiers: on était
  alors utilisateur et on se moquait bien du détail de la structure
  \inlinec{FILE} du moment qu'on avait un descriptif des fonctions
  manipulant cette structure.}). On crée donc un fichier \shell{bst.h}
(l'interface avec l'utilisateur) contenant ce qui suit:
\begin{lstlisting}
#ifndef BST_H
#define BST_H

// bst_t types finite sets of integers with relatively fast searching,
// adding and removing.
typedef struct bst_node_s* bst_t;

// some prototypes incoming...

#endif
\end{lstlisting}
L'usage des macros sera expliquée par la suite, faisons en abstraction
pour le moment. Ce fichier déclare donc qu'il y a un type
\inlinec{struct bst_node_s} et qu'on défini \inlinec{bst_t} comme un
raccourci pour désigner \inlinec{struct bst_node_s*}. Et c'est tout!
L'utilisateur n'a donc accès qu'à ce type \inlinec{bst_t}, censé
représenter les ensembles finis d'entiers. N'oubliez pas d'inclure
\shell{bst.h} dans \shell{bst.c}, ainsi que dans le fichier \shell{.c}
utilisant la bibliothèque: \inlinec{#include "bst.h"}.

\begin{exercise}
  Compléter \inlinec{bst.c} avec les fonctions de bases vues en cours
  manipulant les arbres binaires de recherche: insertion d'un élément,
  suppression d'un élément, recherche d'un élément, recherche du
  maximum, recherche du minimum.

  On ajoutera aussi les fonctions suivantes, simplifiant la vie à
  l'utilisateur:
\begin{lstlisting}
// creates a void BST
bst_t bst_init();

// destructs a BST (freeing any allocated memory)
void bst_destruct(bst_t root); 

// creates a BST containing only a root; in particular, it should 
// allocate sufficient memory on the heap
bst_t bst_sing(int key);

// prints out the BST
void bst_print (bst_t root);
\end{lstlisting}
\end{exercise}

\begin{exercise}
  Rajouter la possibilité de supprimer un élément d'un arbre binaire
  de recherche. On rappelle que pour cela, il est nécessaire d'avoir
  une fonction supprimant le minimum d'un sous-arbre.
\end{exercise}

\section{Plus long sous-mot commum}
\label{sec:lcs}

Rappelons qu'un alphabet $\Sigma$ est juste un ensemble fini dont on
appelle les éléments {\em lettres}. Un mot $w$ sur un alphabet
$\Sigma$ est une séquence finie $(a_1,\dots,a_n)$ avec tous les
$a_i \in \Sigma$. On ommettra les parenthèses et les virgules, non
nécessaires, et on notre $w = a_1\dots a_n$. La longueur d'un mot est
simplement le nombre de lettres qui le composent. L'ensemble de tous
les mots sur un alphabet $\Sigma$ est noté $\Sigma^\ast$.

Un sous-mot d'un mot $w = a_1\dots a_n \in \Sigma^\ast$ est un mot
$u \in \Sigma^\ast$ tel qu'il existe des indices
$1\leq i_1<\dots<i_k \leq n$ tels que $u = a_{i_1} \dots a_{i_k}$. En
français, c'est dire qu'en supprimant certaines lettres de $w$, on
tombe sur $u$. On s'intéresse ici à développer un algorithme renvoyant
un sous-mot le plus grand possible commun à deux mots
$w,w' \in \Sigma^\ast$ donnés.

En écrivant $w = a_1 \dots a_n$ et $w' = b_1 \dots b_m$ (avec
$a_i,b_j \in \Sigma$), on peut caractériser les sous-mots communs de
taille maximale comme suit:
\begin{itemize}
\item soit $n$ ou $m$ est nul et le seul sous-mot commun est le
  séquence vide $\varepsilon$ de longueur $0$,
\item soit $a_n = b_m = x$, et tout sous-mot commun maximal $u$ de
  $a_1\dots a_{n-1}$ et $b_1\dots b_{m-1}$ donne un sous-mot commun
  maximal $ux$ de $w$ et $w'$,
\item soit $a_n\neq b_m$ et un sous-mot commun maximal de $w$ et $w'$
  est un sous-mot maximal commun de $w$ et $b_1\dots b_{m-1}$ ou de
  $a_1\dots a_{n-1}$ et $w'$.
\end{itemize}

\begin{exercise}
  En déduire un algorithme récursif donnant la longueur d'un sous-mot
  maximal de deux mots donnés $w$ et $w'$.

  Cet algorithme est-il efficace? Pourquoi?
\end{exercise}

\medskip

Pour éviter de répéter des calculs déjà effectués, on va utiliser une
technique de {\em programmation dynamique} appelée {\em
  memoisation}. Elle consiste simplement à stocker les valeurs déjà
calculées d'une fonction, afin qu'à un appel ultérieure, on puisse
retourner directement la valeur au lieu de la recalculer bêtement.
\begin{exercise}
  \'Ecrire une fonction:
\begin{lstlisting}
int memoized_lcs (char *v, char *w, int i, int j, int **computed); 
\end{lstlisting}
  prenant en paramètres deux mots $v=v_1\dots v_{n}$ et
  $w = w_1\dots w_{m}$, deux indices $i$ et $j$, et une matrice
  d'entiers \inlinec{computed} et renvoyant la longueur d'un plus long
  sous-mot commun ({\em Longest Common Subsequence} en anglais) de
  $v_1\dots v_i$ et $w_1\dots w_j$. On suppose que la matrice
  \inlinec{computed} est de taille $(n+1) \times (m+1)$ (avec $n$ la
  taille de $v$ et $m$ celle de $w$) et contient en case $(i,j)$
  \begin{itemize}
  \item soit la longueur, déjà calculée, d'un plus long sous-mot
    commun à $v_1\dots v_i$ et $w_1\dots w_j$,
  \item soit $-1$ si la valeur n'a pas encore été calculée.
  \end{itemize}

  En déduire une fonction efficace:
\begin{lstlisting}
int lcs (char *v, char *w);
\end{lstlisting}
  renvoyant la longueur d'un plus long sous-mot commun à $v$ et $w$.
\end{exercise}

\begin{exercise}
  Améliorer les fonctions précédentes pour retourner effectivement un
  sous-mot maximal commun de $v$ et $w$, et non plus seulement sa
  longueur.
\end{exercise}

\section{Pour le 29 novembre}
\label{sec:homeworks}

\begin{exercise}
  Dans la fonction \inlinec{memoized_lcs}, utilise-t-on toutes les
  cases de la matrice?

  Modifier la fonction pour stocker les valeurs déjà calculées non
  plus dans une matrice mais dans un arbre binaire de recherche
  (utilisez une version légèrement modifiée de votre
  bibliothèque). Ainsi, on paye un peu plus cher l'accès aux valeurs
  calculées ($O(\ln n)$ pour $n$ valeurs calculées contre $O(1)$ dans
  le cas de la matrice) mais on économise de l'espace mémoire sinon
  gaspillé\footnote{On pourrait croire que cela est inutile à l'heure
    actuelle, avec les RAM importantes dont l'on dispose. Mais il ne
    faut pas oublier que nos exemples sont ici des ``jouets'' comparés
    à l'utilisation qui est réellement faite des algorithmes ici
    étudiés. D'autre part, on peut aussi penser à l'implémentation de
    ces algorithmes sur des systèmes embarqués disposant de très peu
    de mémoire vive (système de bord d'une voiture, montre
    intelligente, robot, etc.).}.
\end{exercise} 

\end{document}
