% !TEX program = lualatex
\documentclass[french,a4paper]{article}

\usepackage{amsmath,amsthm,amssymb}
\usepackage{fontspec}
% \usepackage{unicode-math}
%\defaultfontfeatures{Ligatures=TeX}
\setmainfont[Ligatures=TeX]{Cambria}
\setsansfont[Ligatures=TeX]{Calibri}
%\setmathfont{Cambria Math}
\setmonofont{Consolas}[Scale=MatchLowercase]

\usepackage{stmaryrd}
\usepackage{polyglossia}
\usepackage{enumitem}
\usepackage{tikz,tikz-cd}
\usepackage[explicit]{titlesec}
\usepackage{listings}
\usepackage{hyperref}
\usepackage{fmtcount}
\usepackage{microtype} % uncomment for better rendering

\setmainlanguage{french}

\title{\sffamily TP3: fonctions, récursivité, complexité}%
\date{27 septembre 2016}%
\author{Algorithmique et projet de programmation --- M1
  mathématiques}%

\lstdefinestyle{sh}{%
  language=sh,%
  basicstyle=\tt\small,%
}
\lstdefinestyle{C}{%
  language=C,%
  breaklines=true,%
  frame=l,%
  xleftmargin=\parindent,%
  basicstyle=\ttfamily\small,%
  keywordstyle=\bfseries\color{green!40!black!80},%
  showstringspaces=false,%
  commentstyle=\itshape\color{purple!70},%
  identifierstyle=\color{blue!80},%
  stringstyle=\color{red!80},%
  directivestyle=\color{orange!90!black!80},%
  upquote=true,%
  otherkeywords={fun_t,arg1_t,arg2_t,typ},%
  escapeinside={<latex>}{</latex>},%
}
\lstset{style=C}

\theoremstyle{definition}
\newtheorem{exercise}{Exercice}
\theoremstyle{remark}
\newtheorem*{remark}{Remarque}

\newcommand{\shell}[1]{\lstinline[style=sh]|#1|}
\newcommand{\inlinec}[1]{\lstinline[style=C]°#1°}

\begin{document}
\maketitle

On s'appliquera, dans tous les exercices, à estimer la complexité des
algorithmes implémentés.

\medskip

Le prototype d'une fonction {\tt C} est la donnée de son type de
retour, de son nom et du type de chacun de ses arguments:
\begin{lstlisting}
fun_t fun_name (arg1_t, arg2_t, ...);
/* or */
fun_t fun_name (arg1_t var1, arg2_t var2, ...);
\end{lstlisting}

Une bonne pratique de programmation est de déclarer tous les
prototypes des fonctions présentes dans un fichier {\tt C} en tête de
celui-ci. Cela permet également d'utiliser les différentes fonctions
avant même leurs déclarations. Par exemple: \lstinputlisting{proto.c}


\section{Récursivité: premier pas}
\label{sec:rec}

\begin{exercise}[\'Echauffement]
  \'Ecrire une fonction {\bf récursive} de prototype
  \begin{lstlisting}
int gcd (int a, int b);
  \end{lstlisting}
  qui calcule le plus grand diviseur commun des deux entiers $a$ et
  $b$.
\end{exercise}

\begin{exercise}[Inverse modulaire]
  \'Ecrire une fonction {\bf récursive} de prototype
  \begin{lstlisting}
int mod_inverse (int a , int n);
  \end{lstlisting}
  qui retourne l'inverse d'un entier $a$ modulo l'entier $n$ (ou $0$
  si $a$ n'est pas premier à $n$). Pour cela, on s'inspirera de
  l'algorithme d'Euclide étendu qui sert à trouver les coefficients de
  Bézout du couple $(a,n)$, c'est-à-dire des entiers $u$ et $v$ tels
  que $ua+vn = \operatorname{gcd}(a,n)$.
\end{exercise}

\section{Exponentiation}
\label{sec:exp}

\begin{exercise}
  \'Ecrire une fonction (la plus naïve possible) de prototype
  \begin{lstlisting}
int exp_int (int a, int e);
  \end{lstlisting}
  qui retourne la valeur de $a^e$.
\end{exercise}

\begin{exercise}
  Observer que
  \begin{displaymath}
    a^e = \left\{
      \begin{aligned}
        (a^{k})^2 &\quad \text{si $e=2k$}\\
        a(a^{k})^2 &\quad \text{si $e=2k+1$}\\
      \end{aligned}
    \right. .
  \end{displaymath}
  En déduire une fonction (récursive) de prototype
  \begin{lstlisting}
int quick_exp (int a, int e);
  \end{lstlisting}
  qui retourne la valeur de $a^e$.
\end{exercise}

\section{Tests de primalité}
\label{sec:prime-test}

\begin{exercise}
  S'inspirer du code écrit pour \inlinec{quick_exp} pour écrire une
  fonction de prototype
  \begin{lstlisting}
int quick_exp_mod (int a, int e, int n);
  \end{lstlisting}
  qui retourne la valeur de $a^e \pmod n$. En particulier, la valeur
  retournée doit être dans l'intervalle entier $[0,n-1]$.
\end{exercise}

Rappelons rapidement le petit théorème de Fermat: si $p \in \mathbb N$
est premier, alors pour tout
$a \in (\mathbb Z/p\mathbb Z) \setminus \{0\}$, on $a^{p-1} = 1$. La
contraposée de ce théorème nous donne donc une manière d'être sûr
qu'un entier $n$ n'est pas premier: il suffit de trouver un nombre
$0<a<n$ tel que $a^{n-1} \neq 1 \pmod n$. On appelle cela le test de
primalité de Fermat.

\begin{exercise}
  \'Ecrire une fonction
  \begin{lstlisting}
char fermat_test(int n, int k);
  \end{lstlisting}
  qui pratique $k$ fois le test de Fermat avec un nombre $0<a<n$
  aléatoire et qui retourne \inlinec{1} si le nombre $n$ est composé
  et \inlinec{0} sinon.
\end{exercise}

Ainsi, si \inlinec{fermat_test(n,k)} renvoie \inlinec{1}, on sait sans
aucun doute que $n$ est composé. \`A l'inverse, si
\inlinec{fermat_test(n,k)} renvoie \inlinec{0}, on ne peut pas
conclure que $n$ est premier: peut-être est-ce un coup de malchance
qu'on ait tiré aucun $a$ tel que $a^{n-1} \neq 1 \pmod n$. On dira
alors que le nombre est {\em probablement premier au sens de Fermat}.

Une idée pour augmenter s'assurer qu'un nombre probablement premier
l'est réellement serait d'augmenter le nombre $k$ de
tirages. Malheureusement, même en testant tous les tirages possibles,
on ne s'en sort pas: le test de Fermat n'est pas une caractérisation
des nombres composés. C'est-à-dire qu'il existe des nombres $n$ {\bf
  composés} pour lesquels tout $a\in [1,n-1]$ premier à $n$ satisfait
à l'équation $a^{n-1} = 1 \pmod n$. De tels nombres sont appelés {\em
  nombres de Carmichael}. Ainsi, si vous utilisez la fonction
\inlinec{fermat_test} sur un nombre de Carmichael (par exemple 561),
vous aurez beau augmenter le nombre $k$ de tests, l'algorithme
répondra à coup sûr \inlinec{0}, vous laissant croire que le nombre
est très probablement premier alors que ce n'est pas le cas!

\medskip

Le test de primalité de Miller-Rabin raffine le test de Fermat. Il se
base sur le fait que dans le corps $\mathbb Z/p\mathbb Z$ (pour $p$
premier), les seules racines carrées de $1$ sont $1$ et $-1$. Ainsi,
devant la conclusion du théorème de Fermat $a^{p-1} = 1 \pmod p$ pour
un entier $0<a<p$, on peut prendre des racines carrées successives
jusqu'à ou bien avoir un exposant impair, ou bien tomber sur
$-1$. Plus formellement, si $p \in \mathbb N$ est premier, on écrit
(de façon unique) $p-1 = 2^sd$ avec $d$ impair, et alors pour tout
entier $a \in [1,p-1]$,
\begin{displaymath}
  a^d = 1 \pmod p
  \quad\text{ou}\quad
  \exists r \in [0,s-1],\, a^{2^rd} = -1 \pmod p
\end{displaymath}
Bien sûr, cette propriété de tient plus nécéssairement si $p$ n'est
pas premier, et tout nombre $a$ satisfaisant sa négation est appelé un
témoin de Miller-Rabin pour $p$.

\begin{exercise}
  \'Ecrire une fonction de prototype
  \begin{lstlisting}
char miller_rabin_test(int n, int k)
  \end{lstlisting}
  qui effectue le test de Miller-Rabin $k$ fois avec un nombre $0<a<n$
  tiré au hasard et retourne \inlinec{1} si $n$ est composé et
  \inlinec{0} sinon.
\end{exercise}

L'avantage du test de Rabin-Miller est qu'il n'y a pas d'équivalent
des nombres de Carmichael. Mieux, on a la proposition suivante: pour
un nombre impair composé $n$, $3/4$ au moins des entiers $1 < a < n$
sont des témoins de Miller-Rabin pour $n$.
\begin{exercise}
  En considérant qu'on tire les entiers de façon uniforme et
  indépendante dans la fonction \inlinec{miller_rabin_test}, majorer
  la probabilité (en fonction de $k$) que la fonction
  \inlinec{miller_rabin_test(n,k)} renvoie \inlinec{0} pour un nombre
  impair composé $n$.
\end{exercise}

\section{Représentation binaire}
\label{sec:binary}

La décomposition d'un entier $n\in \mathbb N$ en base
$b \in \mathbb N \setminus{0}$ is the unique finite sequence
$(n_0,\dots,n_k)$ such that
\begin{displaymath}
  n = \sum_{i=0}^k n_ib^i
  \quad\text{et}\quad
  \forall i \in \mathbb N,\, 0\leq n_i < b.
\end{displaymath}
Cette décomposition s'obtient facilement par division euclidienne
successive de $n$ par $b$.

\medskip

Le but va être de coder des fonctions relatives au changement de
bases. Pour cela, il faut pouvoir implémenter les séquences finies en
{\tt C}: on utilise le concept de {\em tableau}. Un tableau $t$ de
longueur $\ell$ contenant des éléments de type \inlinec{typ} est plus
ou moins une section de $\ell$ cases mémoires adjacentes chacune de la
taille nécessaire pour stocker une variable de type \inlinec{typ}. La
variable $t$ est alors un {\em pointeur} sur la première case de cette
section. Pour accéder aux autres cases, il suffit de ``sauter'' de
\inlinec{sizeof(typ)} en \inlinec{sizeof(typ)} à partir de cette
première case\footnote{Tout ceci peut sembler bien abstrait pour le
  moment, vous deviendrez plus à l'aise avec ces notions d'ici
  quelques semaines. Pour le moment, laissez-vous guider par les
  exemples et appliquez les ``recettes'' données en
  TPs.}. Heureusement, le langage {\tt C} met en place des outils pour
faire tout cela assez facilement. Un tableau \inlinec{t} d'éléments de
type \inlinec{typ} et de longueur $\ell$ est simplement déclaré par
\inlinec{typ t[}$\ell$\inlinec{]}, où $\ell$ doit être un nombre
effectif (et non pas une variable de type \inlinec{int} ou autre). On
accède à la case numéro $i$ avec la syntaxe \inlinec{t[i]}. Attention,
les cases d'un tableau sont numérotées à partir de $0$ en {\tt C}, ce
qui veut dire que $i$ doit être compris entre $0$ et
$\ell-1$. Remarquer qu'un tableau est modifiable {\em en place},
c'est-à-dire qu'une fonction peut modifier ses valeurs directement.

Considérons l'exemple suivant: \lstinputlisting{array.c}

\begin{exercise}
  \'Ecrire une fonction de prototype
  \begin{lstlisting}
void dec2bin(unsigned int n, char* bin);
  \end{lstlisting}
  qui écrit dans le tableau \inlinec{bin} la décomposition binaire
  (i.e.\ en base $b=2$) de l'entier $n$. Tous les tableaux de votre
  programme auront une taille arbitraire fixée à l'avance de $32$
  cases (justifiez!).
\end{exercise}

\begin{exercise}
  \'Ecrire la fonction inverse
  \begin{lstlisting}
unsigned int bin2dec(char* bin);
  \end{lstlisting}
  qui renvoie l'entier dont \inlinec{bin} est la décomposition
  binaire.
\end{exercise}

\begin{exercise}
  Pour rendre les tests des deux fonctions précédentes plus faciles à
  lire, on peut coder une fonction
  \begin{lstlisting}
void pp_bin (char* bin);
  \end{lstlisting}
  qui affiche à l'écran l'écriture usuelle en binaire du nombre dont
  la décomposition est \inlinec{bin}. Par exemple, l'éxécution
  de:
  \begin{lstlisting}
char bin5[32];
dec2bin(5, bin5);
pp_bin(bin5);
  \end{lstlisting}
  doit afficher à l'écran \shell{101}.
\end{exercise}


\section{\`A rendre pour le 18 octobre 2016}
\label{sec:homeworks}

\begin{exercise}[Représentation binaires des relatifs]
  La définition de la décomposition binaire mise en place ci-dessus
  n'est valable que pour les positifs. Pour coder les entiers
  négatifs, il suffit de donner leur signe et la décomposition binaire
  de leur valeur absolue. Il faut donc trouver un artifice qui
  permette d'incorporer le signe à la décomposition binaire: c'est ce
  que fait la méthode qui suit et qu'on appelle {\em le complément à
    2}.

  Avec des tableaux de $0$ et $1$ de $n$ cases, on peut représenter
  les $2^n$ entiers relatifs contenu dans $[-2^{n-1},2^{n-1}-1]$ où
  l'on décide de représenter les entiers $k \in [0,2^{n-1}-1]$ par la
  représentation binaire de $k$, et les entiers $k \in [-2^{n-1},-1]$
  par la représentation binaire de $2^n+k$. La réprésentation d'un
  nobre $k$ ainsi obtenu est appelée la représentation en complément à
  2 de $k$.

  Pour l'exercice, on fixe le nombre de bits $n=32$.
  \begin{enumerate}[label=(\arabic*)]
  \item \'Ecrire une fonction \inlinec{char is_pos(char* bin)} qui
    renvoie 1 si le nombre représenté par \inlinec{bin} en complément
    à 2 est positif ou nul et 0 s'il est strictement négatif.
  \item \'Ecrire une fonction \inlinec{void opposite(char* bin, char*
      opp)} qui écrit dans \inlinec{opp} la représentation en
    complément à 2 de $-k$, où $k$ est le nombre dont \inlinec{bin}
    est la représentation en complément à 2.
  \end{enumerate}
\end{exercise}

\begin{exercise}
  Le langage {\tt C} permet en fait de manipuler les entiers dans leur
  représentation binaire {\bf directement} sans passer par une
  conversion explicite sous forme de tableau. On dispose de certaines
  opérations {\em bits à bits}:
  \begin{center}
    \begin{tabular}{|c|c|c|}
      \hline%
      {\inlinec{a \& b}} & bitwise AND
      & {\inlinec{5 \& 3} donne \inlinec{1}} \\
      \hline%
      {\inlinec{a | b}} & bitwise OR
      & {\inlinec{5 | 3} donne \inlinec{7}} \\
      \hline%
      {\inlinec{a ^ b}} & bitwise XOR
      & {\inlinec{5 ^ 3} donne \inlinec{6}} \\
      \hline%
      {\inlinec{\~a}} & bitwise NOT
      & {\inlinec{\~22} donne \inlinec{9}} \\
      \hline%
      {\inlinec{a >> n}} & right shift by $n$
      & {\inlinec{22 >> 2} donne \inlinec{5}} \\
      \hline%
      {\inlinec{a \<< n}} & left shift by $n$
      & {\inlinec{22 \<< 1} donne \inlinec{44}} \\
      \hline%
    \end{tabular}
  \end{center}
  Pour comprendre ce qu'il se passe, il suffit d'écrire chaque entier
  sous forme binaire et d'effectuer les opérations bit par bit: par
  exemple l'écriture binaire de $5$ est \shell{101} et celle de $3$
  est \shell{11}, ainsi l'entier \inlinec{5 & 3} a pour écriture
  binaire \shell{1}. Pensez-y comme suit:
\begin{verbatim}
  101 (binary for 5)     101 (binary for 5)     101 (binary for 5)
& 011 (binary for 3)   | 011 (binary for 3)   ^ 011 (binary for 3)
  ---                    ---                    ---
  001 (binary for 1)     111 (binary for 7)     110 (binary for 6)
\end{verbatim}
\begin{verbatim}
~10110 (binary for 22)     10110 >> 2             10110 << 1
 -----                     ----------             ----------
 01001 (binary for 9)      101 (binary for 5)     101100 (binary for 44)
\end{verbatim}
  %
  \begin{enumerate}[label=(\arabic*)]
  \item Persuadez-vous (sur papier) que les opérations \inlinec{\%2},
    \inlinec{/2} et \inlinec{*2} peuvent s'écrire avec les opérations
    bits à bits.
  \item En déduire un algorithme calculant le pgcd de deux entiers
    positifs en exploitant l'observation suivante:
    \begin{displaymath}
      \operatorname{gcd}(a,b) = \left\{
        \begin{aligned}
          2\operatorname{gcd}(a/2,b/2)
          &\quad \text{si $a$ et $b$ pairs} \\
          \operatorname{gcd}(a/2,b)
          &\quad \text{si $a$ pair et $b$ impair} \\
          \operatorname{gcd}(a-b,b)
          &\quad \text{si $a$ et $b$ impairs et $a\geq b$} \\
        \end{aligned}
      \right.
    \end{displaymath}
  \end{enumerate}
\end{exercise}

\end{document}