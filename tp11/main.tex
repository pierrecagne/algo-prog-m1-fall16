% !TEX program = lualatex
\documentclass[french,a4paper]{article}

\usepackage{amsmath,amsthm,amssymb}
\usepackage{fontspec}
% \usepackage{unicode-math}
%\defaultfontfeatures{Ligatures=TeX}
\setmainfont[Ligatures=TeX]{Cambria}
\setsansfont[Ligatures=TeX]{Calibri}
%\setmathfont{Cambria Math}
\setmonofont{Consolas}[Scale=MatchLowercase]

\usepackage{stmaryrd}
\usepackage{polyglossia}
\usepackage{enumitem}
\usepackage{tikz,tikz-cd}
\usepackage[explicit]{titlesec}
\usepackage{listings}
\usepackage{hyperref}
\usepackage{fmtcount}
%\usepackage{microtype} % uncomment for better rendering

\setmainlanguage{french}

\title{\sffamily TP11: graphes}%
\date{13 décembre 2016}%
\author{Algorithmique et projet de programmation --- M1
  mathématiques}%

\lstdefinestyle{sh}{%
  language=sh,%
  basicstyle=\tt\small,%
  deletekeywords={hash,type},%
}
\lstdefinestyle{C}{%
  language=C,%
  %breaklines=true,%
  frame=l,%
  xleftmargin=\parindent,%
  basicstyle=\ttfamily\small,%
  keywordstyle=\bfseries\color{green!40!black!80},%
  showstringspaces=false,%
  commentstyle=\itshape\color{purple!70},%
  identifierstyle=\color{blue!80},%
  stringstyle=\color{red!80},%
  directivestyle=\color{orange!90!black!80},%
  upquote=true,%
  morekeywords={typ},%
  escapeinside={<latex>}{</latex>},%
}
\lstset{style=C}

\theoremstyle{definition}
\newtheorem{exercise}{Exercice}
\theoremstyle{remark}
\newtheorem*{remark}{Remarque}

\newcommand{\shell}[1]{\lstinline[style={},style=sh]|#1|}
\newcommand{\inlinec}[1]{\lstinline[style=C]°#1°}

\begin{document}
\maketitle

Dans ce TP, on s'intéresse aux graphes finis simples, c'est-à-dire aux
couples $(V,E)$ avec $V$ un ensemble fini et $E \subseteq V\times V$
une relation binaire anti-réflexive et symétrique. On nomme les
éléments de $V$ des {\em sommets} et ceux de $E$ des {\em arêtes}.

Pour représenter un graphe $(V,E)$ de sommets $v_0,\dots,v_{n-1}$, on
peut utiliser
\begin{itemize}
\item une matrice $A = (a_{i,j})_{0\leq i,j\leq n-1}$, dite
  {\em matrice d'adjacence}, définie par:
  \begin{displaymath}
    a_{i,j} = 
    \left\{
      \begin{aligned}
        1 \quad &\text{si } (v_i,v_j) \in E\\
        0 \quad &\text{sinon}
      \end{aligned}
    \right.
  \end{displaymath}
\item ou un vecteur $t = (t_i)_{0\leq i \leq n-1}$ d'ensembles finis
  défini par:
  \begin{displaymath}
    t_i = \{ j : (v_i,v_j) \in E \}
  \end{displaymath}
\end{itemize}

Pour représenter les ensembles finis du vecteur d'adjacence, on va
utiliser une structure inspirée des listes chainées.
\begin{exercise}
  \'Ecrire une petite bibliothèque (création, ajout, suppression,
  impression, etc.) portant sur les {\em listes circulaires doublement
    chainées}, i.e.
\begin{lstlisting}
typedef struct node_s* cdll; // circular doubly linked list

struct node_s {
  int key;
  cdll left, right;
};
\end{lstlisting}
\end{exercise}

% \begin{exercise}
%   \'Ecrire la matrice d'adjacence du graphe suivant:
%   \begin{center}
%     \begin{tikzpicture}[%
%       every node/.style = {%
%         draw = black,%
%         circle%
%       }%
%       ]%
%       \node (A) at (0,0) {$v_0$};
%       \node (B) at (2,1) {$v_1$};
%       \node (C) at (2,-1) {$v_2$};
%       \node (D) at (4,0) {$v_3$};
%       \draw (A) to (B);
%       \draw (B) to (C);
%       \draw (A) to (C);
%       \draw[bend right=5] (A) to (D);
%       \draw (C) to (D);
%     \end{tikzpicture}
%   \end{center}
%   Que remarquez-vous? Pouviez-vous le prévoir?
% \end{exercise}
\begin{exercise}
  \'Ecrire une fonction \inlinec{cdll* mat_to_vec (int** adj, int n)}
  qui prend en paramètres une matrice d'adjacence d'un graphe et son
  nombre \inlinec{n} de sommets, et qui retourne le vecteur
  d'adjacence de ce même graphe.
\end{exercise}

\medskip

On va implémenter l'agorithme d'Hierholzer qui détermine un {\em
  circuit eulérien} dans un graphe quand il existe. On rappelle qu'un
circuit eulérien est un chemin du graphe partant d'un sommet $v$,
finissant au sommet $v$ et passant par chaque arête une et une seule
fois.

\begin{exercise}
  Montrer qu'un graphe fini simple admet un circuit eulérien si et
  seulement si il est connexe et le degré de tous ses sommets est pair.
\end{exercise}

L'algorithme d'Hierholzer fonctionne sur un graphe connecté avec
sommets de degré pair comme suit:
\begin{itemize}
\item choisir un sommet $v$ quelconque et trouver un circuit simple
  basé en $v$, i.e.\ un chemin de $v$ à $v$ utilisant une arête au
  plus une fois,
\item tant que le circuit construit n'est pas eulérien, choisir un
  sommet $u$ le long du circuit ayant des arêtes adjacentes non
  parcourue et trouver un circuit simple basé en $u$ n'utilisant que
  des arêtes alors non parcourue ; puis combiner ce circuit au circuit
  basé en $v$.
\end{itemize}
Le circuit devient ultimement eulérien, et l'algorithme s'arête alors.

On représentera en {\tt C} un circuit d'un graphe comme la liste
circulaire doublement chainée (\inlinec{cdll}) des sommets parcourus.

\begin{exercise}
  \'Ecrire une fonction
\begin{lstlisting}
cdll cycle_from (int i, int n, cdll* unused, cdll rvertex);
\end{lstlisting}
  qui trouve un circuit simple basé en $i \in \{0,\dots,n-1\}$ dans un
  graphe à $n$ sommets. Le tableau \inlinec{unused} contient en case
  $j$ les listes des arêtes restantes à emprunter depuis $j$: toute
  arête choisie doit en particulier en être retirée. On maintiendra
  également la liste \inlinec{rvertex} des sommets ayant encore des
  arêtes adjacentes non visitées.
\end{exercise}

\begin{exercise}
  En déduire une implémentation de l'algorithme de Hierholzer:
\begin{lstlisting}
cdll euler_cycle (cdll* graph, int n);
\end{lstlisting}
\end{exercise}

\end{document}
