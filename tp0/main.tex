% !TEX program = lualatex
\documentclass[10pt,french,a4paper]{article}

\usepackage{amsmath,amsthm,amssymb}
\usepackage{fontspec}
% \usepackage{unicode-math}
\defaultfontfeatures{Ligatures=TeX}
\setmainfont{XCharter}
\setsansfont{Libertinus Sans}
%% \setmathfont{Libertinus Math}
\setmonofont{Source Code Pro}[Scale=MatchLowercase]

\usepackage{stmaryrd}
\usepackage{polyglossia}
\usepackage{enumitem}
\usepackage{tikz,tikz-cd}
\usepackage[explicit]{titlesec}
\usepackage{listings}
\usepackage{hyperref}
\usepackage{fmtcount}
% \usepackage{microtype} % uncomment for better rendering

\setmainlanguage{french}

\title{\sffamily TP0: découverte de l'environnement}
\date{13 septembre 2016}
\author{Algorithmique et projet de programmation --- M1 mathématiques} 

\lstdefinestyle{sh}{%
  language=sh,%
  basicstyle=\tt\small,%
}
\lstdefinestyle{C}{%
  language=C,%
  basicstyle=\tt\small,%
}
\lstset{style=C}

\theoremstyle{definition}
\newtheorem{exercise}{Exercice}
\theoremstyle{remark}
\newtheorem*{remark}{Remarque}

\newcommand{\shell}[1]{\lstinline[style=sh]|#1|}

\begin{document}
\maketitle

Les énoncés (et quelques corrections) des TPs seront mis au fur et à
mesure sur ma page web, à l'adresse:
\begin{center}
  \url{http://www.normalesup.org/~cagne/teaching.html}
\end{center}
Les exercices du contrôle continu et le projet de programmation final
seront à envoyer à l'adresse email
\href{mailto:pierre.cagne@normalesup.org}
{\nolinkurl{pierre.cagne@normalesup.org}}.

\section{UNIX et la console}
\label{sec:unix}

Les machines de l’université utilisent un système d’exploitation
UNIX-like tel que GNU/Linux, BSD, Solaris, etc. connu pour sa
puissante console. Il s’agit d’un programme permettant à l’utilisateur
de réaliser toutes les tâches administratives sur une machine grâce à
de simples commandes entrées au clavier. En particulier, pour les TPs
de ce cours, on utilisera la console pour compiler nos programmes C.
Toute commande possède un manuel accesible en tapant \shell{man
  [command name]}. Par exemple, pour savoir comment utiliser la
commande \shell{ls}, on peut entrer \shell{man ls}.
\begin{exercise}
  Consulter les manuels des commandes \shell{ls}, \shell{mkdir} et
  \shell{wget}.
\end{exercise}

\begin{exercise}
  En {\bf ligne de commande}, créer un dossier \shell{td0} et y
  télécharger le fichier
  \url{http://www.normalesup.org/~cagne/tp0.zip}. Extraire le fichier
  téléchargé à l'aide de la commande \shell{unzip}.
\end{exercise}

\begin{remark}
  Se référer aux pages \shell{man} doit devenir un rélfexe quand vous
  n'êtes pas sûr de l'utilisation d'une commande. C'est l'esprit
  RTFM...
\end{remark}

\section{Le langage C}
\label{sec:c-lang}

Le langage \shell{C} est un langage de programmation, au même titre
que \shell{Python}, \shell{Java}, \shell{Scilab}, \shell{R}, etc. que
vous avez peut-être déjà utilisé. C'est un ``vieux'' langage (il date
de 1972), ce qui lui fournit à la fois des avantages (large
communauté, très sûr, optimisé pour la plupart des compilateurs, etc.)
et des inconvénients (syntaxe lourde, absence des paradigmes
``nouveaux'', gestion de la mémoire manuelle, etc.). C'est un langage
{\em compilé}, c'est-à-dire qu'il fonctionne approximativement de la
manière suivante:
\begin{itemize}
\item on écrit un fichier texte avec un éditeur tel \shell{emacs},
  avec une syntaxe bien définie (que l'on va apprendre tout au long du
  semestre), dont l'extension doit être \shell{.c},disons
  \shell{main.c} pour l'exemple,
\item on appelle un compilateur, usuellement via la ligne de commande,
  sur le fichier \shell{main.c} nouvellement créé; son rôle est de
  {\em traduire} le code source écrit en \shell{C} en code machine
  compréhensible directement par le processeur; on utilisera dans ce
  cours le compilateur \shell{gcc}, sûrement le plus utilisé au monde,
\item on obtient alors en sortie un fichier exécutable, disons
  \shell{main.out}, q'uon peut alors appeler grâce à la ligne de
  commande (presque) comme n'importe quel autre programme.
\end{itemize}

Plus spéficiquement, pour compiler un fichier source appelé
\shell{main.c} en un programme exécutable \shell{main.out}, on se
place dans le dossier contenant le source et on exécute la commande
suivante:
\begin{lstlisting}[style=sh]
  gcc -Wall -o main.out main.c
\end{lstlisting}
On voit que \shell{gcc} prend ici, outre le source \shell{main.c},
deux autres arguments commençant par des tirets. Ce sont des {\em
  flags} qui agissent sur le comportement de \shell{gcc}. Le premier,
\shell{-Wall}, signifie {\em warnings all} et permet d'afficher toutes
sortes d'avertissements de la part du compilateur: vous êtes libre de
ne pas utiliser ce flag, mais vous risquez alors d'avoir plus de mal à
comprendre les erruers de compilations qui pouraient surgir. Le
deuxième flag est \shell{-o}, signifiant {\em out}, et prend lui-même
en paramètre un nom de fichier, ici \shell{main.out}: il sert à
indiquer le nom du fichier exécutable créé par le compilateur à partir
du source \shell{main.c}: on peut également ommettre ce flag, mais
alors l'exécutable obtient par défaut le nom très explicite de
\shell{a.out}...

On peut maintenant exécuter notre programme simplement en rentrant son
chemin d'accès. Attention, si on l'exécute depuis le dossier le
contenant (et c'est le cas le plus fréquent), il faut entrer
\shell{./main.out} et non pas seulement \shell{main.out}.

\begin{remark}
  L'extension en \shell{.out} n'a rien d'obligatoire. En fait, la
  norme est plutôt de ne pas utiliser d'extension du tout: pour ouvrir
  Mozilla Firefox par exemple, on utilise en effet la commande
  \shell{firefox} et non pas \shell{firefox.out}!
\end{remark}

\begin{exercise}
  Compiler le fichier \shell{hello.c} (présent parmi les fichiers
  extraits précédemment) et exécuter le programme obtenu.
\end{exercise}

\begin{exercise}
  Ouvrir le fichier \shell{hello.c} depuis la ligne de commande. Par
  exemple, si on utilise l'éditeur \shell{emacs}, on exécute la
  commande:
  \begin{lstlisting}[style=sh]
    emacs hello.c &
  \end{lstlisting}
  (Le symbole \shell{&} donne l'ordre au terminal de nous rendre la
  main avant la fin du processus lancé. En effet, on voudrait pouvoir
  réutiliser la ligne de commande sans avoir à fermer notre éditeur de
  texte!)

  Essayer de comprendre la syntaxe du document.
\end{exercise}

\begin{exercise}
  Modifier le fichier \shell{hello.c} et le recompiler pour voir
  l'effet de la modification sur l'exécution du programme.
\end{exercise}

\begin{exercise}
  Faire de même avec les autres fichiers extraits de \shell{tp0.zip}.
\end{exercise}

\end{document}