% !TEX program = lualatex
\documentclass[french,a4paper]{article}

\usepackage{amsmath,amsthm,amssymb}
\usepackage{fontspec}
% \usepackage{unicode-math}
%\defaultfontfeatures{Ligatures=TeX}
\setmainfont[Ligatures=TeX]{Cambria}
\setsansfont[Ligatures=TeX]{Calibri}
%\setmathfont{Cambria Math}
\setmonofont{Consolas}[Scale=MatchLowercase]

\usepackage{stmaryrd}
\usepackage{polyglossia}
\usepackage{enumitem}
\usepackage{tikz,tikz-cd}
\usepackage[explicit]{titlesec}
\usepackage{listings}
\usepackage{hyperref}
\usepackage{fmtcount}
%\usepackage{microtype} % uncomment for better rendering

\setmainlanguage{french}

\title{\sffamily TP10: programmation dynamique}%
\date{6 décembre 2016}%
\author{Algorithmique et projet de programmation --- M1
  mathématiques}%

\lstdefinestyle{sh}{%
  language=sh,%
  basicstyle=\tt\small,%
  deletekeywords={hash,type},%
}
\lstdefinestyle{C}{%
  language=C,%
  %breaklines=true,%
  frame=l,%
  xleftmargin=\parindent,%
  basicstyle=\ttfamily\small,%
  keywordstyle=\bfseries\color{green!40!black!80},%
  showstringspaces=false,%
  commentstyle=\itshape\color{purple!70},%
  identifierstyle=\color{blue!80},%
  stringstyle=\color{red!80},%
  directivestyle=\color{orange!90!black!80},%
  upquote=true,%
  morekeywords={typ},%
  escapeinside={<latex>}{</latex>},%
}
\lstset{style=C}

\theoremstyle{definition}
\newtheorem{exercise}{Exercice}
\theoremstyle{remark}
\newtheorem*{remark}{Remarque}

\newcommand{\shell}[1]{\lstinline[style={},style=sh]|#1|}
\newcommand{\inlinec}[1]{\lstinline[style=C]°#1°}

\begin{document}
\maketitle

\section{Problème du sac à dos}
\label{sec:knapsack}

Vous êtes un voleur cambriolant une bijouterie. Vous disposez d'un sac
d'une contenance $W$ et vous connaissez le poids $w_i$ et la valeur
$v_i$ de chacun des $i\in \{1,\dots, n\}$ objets présents dans la
bijouterie. Votre but est de maximiser le butin que vous pouvez
transporter dans votre sac.

Pour ce faire, on va invoquer les techniques de programmation
dynamique.

\begin{exercise}
  Notons $m_{i,w}$ le butin maximal que l'on peut obtenir avec un sac
  de contenance $w$ et en ne regardant que les objets $1,\dots,i$.

  Exprimer $m_{i,w}$ en fonction de $m_{i-1,w}$, $m_{i-1,w-w_i}$ et
  $v_i$ en considérant la dichotomie suivante:
  \begin{itemize}
  \item soit $w_i > w$, auquel cas on ne peut même pas prendre l'objet
    $i$ dans le sac,
  \item soit $w_i \leq w$ et il faut déterminer si prendre l'objet $i$
    est plus rentable que de le laisser sur place.
  \end{itemize}
\end{exercise}

\begin{exercise}
  En déduire un algorithme récursif naïf calculant la valeur
  souhaitée, à savoir $m_{n,W}$. (Il n'est pas nécéssaire de
  l'implémenter en {\tt C}.)
\end{exercise}

On effectue maintenant l'optimisation propre à la programmation
dynamique, la mémoization.

\begin{exercise}
  \'Ecrire une fonction {\tt C} de prototype:
\begin{lstlisting}
int memoized_knapsack (int *values, int *weights, int i, int w,
                       int **computed);
\end{lstlisting}
  Elle prendra en paramètres un tableau \inlinec{values} contenant en
  case $j$ la valeur $v_j$, un tableau \inlinec{weights} contenant en
  case $j$ le poids $w_j$, deux index $i$ et $w$ et une matrice
  \inlinec{computed}. Cette dernière contient en case $(i,w)$
  \begin{itemize}
  \item la valeur $m_{i,w}$ définie précédemment si elle a déjà été calculée,
  \item la valeur $-1$ sinon.
  \end{itemize}
  La fonction doit renvoyer l'entier $m_{i,w}$.
\end{exercise}

Il s'agit maintenant de mettre en place la deuxième technique propre à
la programmation dynamique: se «~laisser des indices~» lors du
remplissage du tableau \inlinec{computed} afin de pouvoir remonter les
choix faits lors de la résolution récursive.

\begin{exercise}
  Modifier légèrement la fonction précédente de façon à obtenir un
  fonction
\begin{lstlisting}
int memoized_knapsack (int *values, int *weights, int i, int w,
                       int **computed, int **choices);
\end{lstlisting}
  où tous les paramètres restent les même que précédemment et où la
  matrice \inlinec{choices} contient en case $(i,w)$
  \begin{itemize}
  \item soit $0$ si on sait qu'on ne choisit pas l'objet $i$,
  \item soit $1$ si on sait qu'on choisit l'objet $i$,
  \item soit $-1$ si la valeur n'est pas encore calculée.
  \end{itemize}
\end{exercise}

\begin{exercise}
  En déduire une fonction
\begin{lstlisting}
int knapsack (int *values, int *weights, int n, int W);
\end{lstlisting}
  résolvant notre problème de voleur.
\end{exercise}

\end{document}
