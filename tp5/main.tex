% !TEX program = lualatex
\documentclass[french,a4paper]{article}

\usepackage{amsmath,amsthm,amssymb}
\usepackage{fontspec}
% \usepackage{unicode-math}
%\defaultfontfeatures{Ligatures=TeX}
\setmainfont[Ligatures=TeX]{Cambria}
\setsansfont[Ligatures=TeX]{Calibri}
%\setmathfont{Cambria Math}
\setmonofont{Consolas}[Scale=MatchLowercase]

\usepackage{stmaryrd}
\usepackage{polyglossia}
\usepackage{enumitem}
\usepackage{tikz,tikz-cd}
\usepackage[explicit]{titlesec}
\usepackage{listings}
\usepackage{hyperref}
\usepackage{fmtcount}
\usepackage{microtype} % uncomment for better rendering

\setmainlanguage{french}

\title{\sffamily TP5: structures, chaînes de caractères}%
\date{31 octobre 2016}%
\author{Algorithmique et projet de programmation --- M1
  mathématiques}%

\lstdefinestyle{sh}{%
  language=sh,%
  basicstyle=\tt\small,%
}
\lstdefinestyle{C}{%
  language=C,%
  %breaklines=true,%
  frame=l,%
  xleftmargin=\parindent,%
  basicstyle=\ttfamily\small,%
  keywordstyle=\bfseries\color{green!40!black!80},%
  showstringspaces=false,%
  commentstyle=\itshape\color{purple!70},%
  identifierstyle=\color{blue!80},%
  stringstyle=\color{red!80},%
  directivestyle=\color{orange!90!black!80},%
  upquote=true,%
  otherkeywords={},%
  escapeinside={<latex>}{</latex>},%
}
\lstset{style=C}

\theoremstyle{definition}
\newtheorem{exercise}{Exercice}
\theoremstyle{remark}
\newtheorem*{remark}{Remarque}

\newcommand{\shell}[1]{\lstinline[style={},style=sh]|#1|}
\newcommand{\inlinec}[1]{\lstinline[style=C]°#1°}

\begin{document}
\maketitle

\section{Listes chaînées}
\label{sec:linked-list}

Les listes chaînées sont une structure standard de l'algorithmique et
sont complémentaires aux tableaux. Un tableau est une structure
efficace quand on sait à l'avance (une borne sur) le nombre d'éléments
à stocker: on peut alors accéder aux éléments par leur index en
$O(1)$. Ce qui permet une telle efficacité est la contiguité des cases
du tableau en mémoire, ramenant l'accès à la case $i$ à une simple
addition sur les pointeurs. En revanche, ils sont très inefficace
concernant l'ajout ou la suppression d'éléments: pour ajouter un
élément en tête d'un tableau de taille $n$, il faut allouer un tableau
de taille $n+1$, puis copier l'entièreté du tableau dans les $n$
dernières cases avant enfin d'affecter la première case à la valeur à
ajouter... tout cela se fait au mieux en $O(n)$!

Les listes chaînées prennent le point de vue opposées. Au diable la
contiguité en mémoire et l'arithmétique des pointeurs: une case est
maintenant stockée avec un pointeur sur la case suivante. Rajouter une
valeur en tête d'une telle liste est alors un jeu d'enfant: on alloue
une case mémoire où bon nous semble dans la mémoire (le plus souvent
sur le tas) et on lui associe un pointeur vers l'ancienne première
case de la liste. Et tout ça en $O(1)$! Bien entendu, on a perdu
l'accès à la $i$-ème case en temps constant, il faut maintenant partir
du premier élément et suivre les pointeurs successifs jusqu'à
atteindre le $i$-ième élément... Tout dépend donc de ce que l'on veut
faire, il n'y a pas de solution miracle.

\begin{exercise}
  Implémenter une structure $\tt C$ codant les listes châinées
  d'entiers. Indice: au même titre qu'en {\tt C} un tableau d'entiers
  est en fait un pointeur vers le premier entier de ce tableau, une
  liste chaînée d'entiers sera un pointeur vers la première instance
  d'une structure à deux champs:
  \begin{lstlisting}
typedef list struct node *;
    
struct node {
  // two fields      
};
  \end{lstlisting}
 
  \'Ecrire une fonction renvoyant une liste vide:
  \begin{lstlisting}
list new_list ();
  \end{lstlisting}
\end{exercise}

\begin{exercise}
  \'Ecrire une fonction prenant en paramètres une liste et un entier
  et renvoyant la concaténation de cet entier en tête de liste:
  \begin{lstlisting}
list push (list l, int val);
  \end{lstlisting}
\end{exercise}

Si les tableaux étaient adaptés à un traitement itératif, les listes
chaînées, elles, se prêtent plus facilement à un traitement recursif.

\begin{exercise}
  \'Ecrire une fonction calculant la longueur d'une liste:
  \begin{lstlisting}
int length (list l);
  \end{lstlisting}
\end{exercise}

\begin{exercise}
  \'Ecrire une fonction retournant une liste (le dernier élément est
  maintenant le premier, l'avant-dernier le deuxième, etc.):
  \begin{lstlisting}
void reverse (list l);
  \end{lstlisting}
  Attention, cette fonction devra avoir une complexité en temps de
  $O(n)$ (avec $n$ la longueur de la liste).
\end{exercise}

\section{Manipulation de fichiers}
\label{sec:sort-unix}

On va maintenant s'atteler à recoder (une version simple de) la
commande \shell{sort} des sytèmes UNIX. C'est un petit programme qui
prend en paramètre un chemin d'accès à un fichier texte et affiche à
l'écran l'ensemble de ses lignes triées alphabétiquement. Ainsi, si un
fichier \shell{phonebook} contient:
\lstinputlisting[style={},style=sh]{phonebook.txt}
L'exécution de la commande \shell{sort phonebook} affichera:
\begin{lstlisting}[style={},style=sh]
Adrien 0694367833
Bernard 0663896735
Claudia 0642356732
Clement 0678900034
Lea 0612673546
Zaid 0673894765
\end{lstlisting}

\begin{exercise}
  \'Ecrire une fonction qui compare deux chaînes de caractères dans
  l'ordre alphabétique:
  \begin{lstlisting}
char compare_str (char* s, char* t);
  \end{lstlisting}
  Elle renverra \inlinec{1} si \inlinec{s} est inférieur à \inlinec{t}
  au sens large et \inlinec{0} sinon. On rappelle que
  \inlinec{s}${}\leq{}$\inlinec{t} si et seulement si
  \begin{enumerate}[label=-]
  \item \inlinec{s} est un préfixe de \inlinec{t}
  \item il existe $i\geq 0$ tel que \inlinec{s[k] = t[k]} pour tout
    $0\leq k\leq i-1$ et \inlinec{s[i]}${}\leq{}$\inlinec{t[i]}.
\end{enumerate}

\end{exercise}

\begin{exercise}
  \'Ecrire une fonction adaptée des fonctions de tri vues la dernière fois:
  \begin{lstlisting}
void sort_str (char** strs);
  \end{lstlisting}
  qui trie alphabétiquement un tableau de chaînes de caractères.
\end{exercise}

Rappelons que pour traiter les fichiers textes, le langage {\tt C}
dispose des fonction suivantes, disponibles dans la bibliothèque
\inlinec{stdio.h}: \lstinputlisting{fopen.c}

\begin{exercise}
  En déduire un programme implémentant la commande \shell{sort}
  décrite précédemment. En particulier, on déterminera le nombre de
  lignes et la taille maximale des lignes du fichier passé en
  paramètre pour allouer un tableau de taille suffisante.
\end{exercise}

\section{Pour le 7 novembre 2016}
\label{sec:homeworks}

Déterminer des binômes et commencer à regarder le projet. Préparer des
questions si besoin à poser la fois prochaine.

\end{document}