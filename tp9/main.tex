% !TEX program = lualatex
\documentclass[french,a4paper]{article}

\usepackage{amsmath,amsthm,amssymb}
\usepackage{fontspec}
% \usepackage{unicode-math}
%\defaultfontfeatures{Ligatures=TeX}
\setmainfont[Ligatures=TeX]{Cambria}
\setsansfont[Ligatures=TeX]{Calibri}
%\setmathfont{Cambria Math}
\setmonofont{Consolas}[Scale=MatchLowercase]

\usepackage{stmaryrd}
\usepackage{polyglossia}
\usepackage{enumitem}
\usepackage{tikz,tikz-cd}
\usepackage[explicit]{titlesec}
\usepackage{listings}
\usepackage{hyperref}
\usepackage{fmtcount}
%\usepackage{microtype} % uncomment for better rendering

\setmainlanguage{french}

\title{\sffamily TP9: Programmation dynamique et fonction de hashage}%
\date{29 novembre 2016}%
\author{Algorithmique et projet de programmation --- M1
  mathématiques}%

\lstdefinestyle{sh}{%
  language=sh,%
  basicstyle=\tt\small,%
  deletekeywords={hash,type},%
}
\lstdefinestyle{C}{%
  language=C,%
  %breaklines=true,%
  frame=l,%
  xleftmargin=\parindent,%
  basicstyle=\ttfamily\small,%
  keywordstyle=\bfseries\color{green!40!black!80},%
  showstringspaces=false,%
  commentstyle=\itshape\color{purple!70},%
  identifierstyle=\color{blue!80},%
  stringstyle=\color{red!80},%
  directivestyle=\color{orange!90!black!80},%
  upquote=true,%
  morekeywords={typ},%
  escapeinside={<latex>}{</latex>},%
}
\lstset{style=C}

\theoremstyle{definition}
\newtheorem{exercise}{Exercice}
\theoremstyle{remark}
\newtheorem*{remark}{Remarque}

\newcommand{\shell}[1]{\lstinline[style={},style=sh]|#1|}
\newcommand{\inlinec}[1]{\lstinline[style=C]°#1°}

\begin{document}
\maketitle

\section{Plus long sous-mot commum}
\label{sec:lcs}

Rappelons qu'un alphabet $\Sigma$ est juste un ensemble fini dont on
appelle les éléments {\em lettres}. Un mot $w$ sur un alphabet
$\Sigma$ est une séquence finie $(a_1,\dots,a_n)$ avec tous les
$a_i \in \Sigma$. On ommettra les parenthèses et les virgules, non
nécessaires, et on notre $w = a_1\dots a_n$. La longueur d'un mot est
simplement le nombre de lettres qui le composent. L'ensemble de tous
les mots sur un alphabet $\Sigma$ est noté $\Sigma^\ast$.

Un sous-mot d'un mot $w = a_1\dots a_n \in \Sigma^\ast$ est un mot
$u \in \Sigma^\ast$ tel qu'il existe des indices
$1\leq i_1<\dots<i_k \leq n$ tels que $u = a_{i_1} \dots a_{i_k}$. En
français, c'est dire qu'en supprimant certaines lettres de $w$, on
tombe sur $u$. On s'intéresse ici à développer un algorithme renvoyant
un sous-mot le plus grand possible commun à deux mots
$w,w' \in \Sigma^\ast$ donnés.

En écrivant $w = a_1 \dots a_n$ et $w' = b_1 \dots b_m$ (avec
$a_i,b_j \in \Sigma$), on peut caractériser les sous-mots communs de
taille maximale comme suit:
\begin{itemize}
\item soit $n$ ou $m$ est nul et le seul sous-mot commun est le
  séquence vide $\varepsilon$ de longueur $0$,
\item soit $a_n = b_m = x$, et tout sous-mot commun maximal $u$ de
  $a_1\dots a_{n-1}$ et $b_1\dots b_{m-1}$ donne un sous-mot commun
  maximal $ux$ de $w$ et $w'$,
\item soit $a_n\neq b_m$ et un sous-mot commun maximal de $w$ et $w'$
  est un sous-mot maximal commun de $w$ et $b_1\dots b_{m-1}$ ou de
  $a_1\dots a_{n-1}$ et $w'$.
\end{itemize}

\begin{exercise}
  En déduire un algorithme récursif donnant la longueur d'un sous-mot
  maximal de deux mots donnés $w$ et $w'$.

  Cet algorithme est-il efficace? Pourquoi?
\end{exercise}

\medskip

Pour éviter de répéter des calculs déjà effectués, on va utiliser une
technique de {\em programmation dynamique} appelée {\em
  memoisation}. Elle consiste simplement à stocker les valeurs déjà
calculées d'une fonction, afin qu'à un appel ultérieure, on puisse
retourner directement la valeur au lieu de la recalculer bêtement.
\begin{exercise}
  \'Ecrire une fonction:
\begin{lstlisting}
int memoized_lcs (char *v, char *w, int i, int j, int **computed); 
\end{lstlisting}
  prenant en paramètres deux mots $v=v_1\dots v_{n}$ et
  $w = w_1\dots w_{m}$, deux indices $i$ et $j$, et une matrice
  d'entiers \inlinec{computed} et renvoyant la longueur d'un plus long
  sous-mot commun ({\em Longest Common Subsequence} en anglais) de
  $v_1\dots v_i$ et $w_1\dots w_j$. On suppose que la matrice
  \inlinec{computed} est de taille $(n+1) \times (m+1)$ (avec $n$ la
  taille de $v$ et $m$ celle de $w$) et contient en case $(i,j)$
  \begin{itemize}
  \item soit la longueur, déjà calculée, d'un plus long sous-mot
    commun à $v_1\dots v_i$ et $w_1\dots w_j$,
  \item soit $-1$ si la valeur n'a pas encore été calculée.
  \end{itemize}

  En déduire une fonction efficace:
\begin{lstlisting}
int lcs (char *v, char *w);
\end{lstlisting}
  renvoyant la longueur d'un plus long sous-mot commun à $v$ et $w$.
\end{exercise}

\begin{exercise}
  Améliorer les fonctions précédentes pour retourner effectivement un
  sous-mot maximal commun de $v$ et $w$, et non plus seulement sa
  longueur.
\end{exercise}

\section{Table de hashage}
\label{sec:hash-table}

Les tables de hashage sont le pendant informatique des partitions d'un
ensemble. Partitionner un ensemble (fini) $S$ de taille
$\lvert S \rvert$ en $m\in \mathbb N$ sous-ensembles est équivalent à
se donner une fonction $h : S \to \{0,\dots,m-1\}$: en effet, une
telle fonction étiquette chacun des éléments de $S$ avec le numéro $i$
du sous-ensemble $h^{-1}(i)$ de la partition le contenant.

En terme du langage {\tt C}, cela veut dire que si l'on dispose d'une
fonction de la forme \inlinec{int h(typ x)} à valeurs dans
$\{0,\dots,m-1\}$ pour un certain $m \in \mathbb N$, alors on peut
représenter un ensemble fini $S$ d'éléments de type \inlinec{typ}
comme un tableau de taille $m$ où la case indexée par \inlinec{i}
contient les éléments \inlinec{x} de $S$ tels que \inlinec{h(x)}
retourne \inlinec{i}. Encore faut-il trouver un moyen de ``faire
rentrer'' tous ces éléments dans une même case... On peut utiliser
toute les structures de données que vous avez pu voir, dépendamment de
l'application.

Pour ce TP, par souci de simplicité, on choisit d'utiliser des listes
chaînées, c'est-à-dire que chaque case de notre table de hashage
contiendra un pointeur vers le premier noeud d'une liste
chaînée. Enfin, pour fixer les idées, le type abstrait \inlinec{typ}
sera ici le type des couples d'entiers:
\begin{lstlisting}
struct coord_s {
  int x,y;
} coord_t;
\end{lstlisting}

\begin{exercise}[au tableau]
  \'Ecrire des fichiers \shell{coord.c} et \shell{list.c} (et leurs
  headers) permettant de manipuler les couples et les listes chaînées
  de couples.
\end{exercise}

Dans un fichier \shell{hash.c}, on va implémenter la
structure {\tt C} correspondant aux tables de hashage d'éléments de
type \inlinec{coord_t}:
\begin{lstlisting}
struct hash_s {
  int size; // size of table
  list_t *table; // actual hash table
};
\end{lstlisting}
Ici, on veut cacher les détails de l'implémentation à l'utilisateur de
notre bibliothèque \shell{hash.c}. On va donc simplement
lui donner accès à la structure via un pointeur dans \shell{hash.h}:
\begin{lstlisting}
#ifndef HASH_H
#define HASH_H

typedef struct hash_s* hash_t;

// other function declarations

#endif
\end{lstlisting}
Rappelez-vous qu'un tel \inlinec{typedef}, sans definition explicite
de la structure dans le header, doit nécessairement porter sur un
pointeur. Un simple
\begin{lstlisting}
typedef struct hash_s hash_t;

\end{lstlisting}
est prohibé: la déclaration d'une variable de type \inlinec{hash_t}
par l'utilisateur n'ayant accès qu'au header serait alors un vrai
casse-tête pour {\tt C}, qui, ne sachant pas combien de place allouer
sur la pile d'exécution, préfère échouer à la compilation (erreur de
type \shell{incomplete type}). L'allocation d'un pointeur en revanche
ne pose aucun problème à {\tt C}: quelque soit l'objet vers lequel
pointe celui-ci, il n'y a besoin que d'une case mémoire pouvant
contenir une adresse de la RAM!

\begin{exercise}
  Proposer une fonction mathématique
  \begin{displaymath}
    h : \mathbb N\times \mathbb N\to \{0,\dots,m-1\}
  \end{displaymath}
  qui soit une bonne fonction de hashage. L'implémenter en {\tt C}.
\end{exercise}

\begin{exercise}
  Ecrire les fonctions suivantes:
\begin{lstlisting}
// create a (chained) hash table of given size
hash_t hash_init (int size);

// insert key inside htab, with value val
hash_t hash_ins (coord_t key, int val, hash_t htab);

// remove key from htab
hash_t hash_rem (coord_t key, hast_t htab);

// search for key in htab, returns 1 if found, 0 otherwise
char hash_mem (coord_t key, hast_t htab);

// return value contained at key in htab
// Careful: key should be member of htab
int hash_getval (coord_t key, hast_t htab);
\end{lstlisting}
\end{exercise}

\section{Pour le 6 décembre 2016}
\label{sec:homeworks}

\begin{exercise}~%
  \begin{enumerate}[label=(\roman*)]
  \item Modifier les fonctions du premier exercice afin d'afficher la
    matrice \inlinec{computed} après l'appel à \inlinec{memoized_lcs}
    sur la longueur de $v$ et $w$. Que remarquer? (Indice: utiliser de
    grandes chaînes de caractères.)
  \item Utiliser la structure de table de hashage mise au point dans
    l'exercice 2 pour palier à ce problème.
  \end{enumerate}
\end{exercise}

\end{document}
